\section{Introduction}\label{sec:introduction}
Ancient mariners could look up the night sky, point out which stars they were looking at, and navigate across the globe
without the use of maps.
\textit{Star identification algorithm} refers to a computational approach to pointing out which stars
are in the sky.
Given an image of the sky, star identification is matching the bright spots in an image, to stars in an astronomical
catalog.
The device that performs these computations is the star tracker, much like the navigators on the ship.
\textit{Lost-in-space} refers to an additional constraint on the problem: the absence of knowing where we took
the picture and how we pointed the camera.

This problem is most prevalent in designing LEO (low Earth orbit) spacecraft.
In order for a craft to point a payload, direct thrusters, or orient it's solar panels, an accurate
\textit{attitude} (another term for orientation) must be known.
There are a few known landmarks in space where some attitude can be extracted (the Earth, the Sun), but this
requires constant direction towards just these objects.
Star trackers do not limit themselves to a single object, instead they use the entire sky of stars to determine it's
orientation.

There exist roughly 4500 stars in the sky visible to the human eye.
It is computationally expensive to iterate through each combination of stars in the sky, so star identification methods
are used instead.
This paper analyzes six existing methods, all of which involve the following process:
\begin{enumerate}
    \item Given an image of stars.
    \item Identify a select few stars in the image.
    \item Guess how we are oriented.
    \item Identify the rest of the stars in the image.
    \item Finalize and determine our orientation.
\end{enumerate}

Each method's feature uniqueness, permutation order, candidate reduction, and identification process will be
characterized under the introduction of various noise.
The process of identifying blobs in an image, constructing the image coordinate system, and efficiently querying
static databases is not addressed here.

This research is motivated by a growing difference in the number of stellar attitude determination methods and the
comparison between each of these methods in a more controlled manner for star tracker development.
Interchangeable factors are abstracted away (camera hardware, blob detection, etc\ldots) to focus more on how each
method matches stars in an image to a catalog.

This paper aims to contribute a general flow for most star identification methods, a process for comparing each
method, and characterization of several existing methods.
