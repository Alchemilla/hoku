\newcommand{\algorithmautorefname}{Algorithm}
\newcommand{\algeqref}[1]{\hyperref[#1]{Eq.~\ref*{#1}}}
\algnewcommand{\LineComment}[1]{\State \(\triangleright\) #1}
\algrenewcommand\algorithmicindent{0.35cm}
\newcommand{\bigO}{\mathcal{O}}
\newcommand{\abs}[1]{\left| #1 \right|}
\newcommand{\set}[1]{\left\{ #1 \right\}}
\MakeRobust{\Call}

\restylefloat{algorithm}

\begin{figure}[ht]
    \centering{
    % Style for process block.
\tikzstyle{process} = [rectangle, text width=3cm, minimum width=3cm, minimum height=1cm,text centered, draw=black,
fill=orange!30]

% Style for terminal block.
\tikzstyle{terminal} = [rectangle, text width=1.7cm, minimum width=1.7cm, minimum height=1.7cm,text centered,
draw=black, fill=red!30]

% Style for decision block.
\tikzstyle{decision} = [diamond, text width=2cm, minimum width=2.5cm, minimum height=2.5cm,text centered, draw=black,
fill=green!30, inner sep=-10pt]

% Style for line.
\tikzstyle{line} = [draw, -latex']

\begin{tikzpicture}[node distance=1.8cm]
    \node[scale=1](getImage)[terminal]{Get Camera Image};
    \node[scale=1](pickQueryStars) [process, left of=getImage, xshift = -1.7cm] {Pick $k$ Image Stars};
    \node[scale=1](searchCatalog)[process, below of=pickQueryStars] {Search Catalog};
    \node[scale=1](confidentInCatalog)[decision, below of=searchCatalog, yshift=-0.5cm] {$|R| > 0$?};
    \node[scale=1](filterCandidates)[process, below of=confidentInCatalog, yshift=-0.5cm] {Filter Candidates};
    \node[scale=1](confidentAfterFilter)[decision, below of=filterCandidates, yshift=-0.5cm] {Confident in $r$?};
    \node[scale=1](findMap)[process, below of=confidentAfterFilter, yshift=-0.5cm]{Identify};
    \node[scale=1](confidentInMap)[decision, below of=findMap, yshift=-0.5cm] {Confident in $a$?};
    \node[scale=1](returnMap)[terminal, right of=confidentInMap, xshift = 1.7cm] {Return Map};

    \draw[->,>=stealth](getImage) -- node[scale=1.3, yshift=-0.3cm]{$I$}(pickQueryStars);
    \draw[->,>=stealth] (pickQueryStars) -- node[scale=1.3, xshift=0.5cm]{$b$}(searchCatalog);
    \draw[->, >=stealth] (searchCatalog) -- node[scale=1.3, xshift=0.5cm, yshift=-0.15cm]{$R$}(confidentInCatalog);
    \draw[->, >=stealth] (confidentInCatalog) -- node[anchor=east, yshift=0.1cm]{Yes}(filterCandidates);
    \draw[->, >=stealth] (filterCandidates) -- node[scale=1.3, xshift=0.5cm, yshift=-0.15cm]{$r$}(confidentAfterFilter);
    \draw[->, >=stealth] (confidentInMap) -- node[xshift=-0.05cm, yshift=0.2cm]{Yes}(returnMap);
    \draw[->, >=stealth] (confidentAfterFilter) -- node[anchor=east, yshift=0.1cm]{Yes} (findMap);
    \draw[->, >=stealth] (findMap) -- node[scale=1.3, xshift=0.5cm, yshift=-0.15cm]{$a$}(confidentInMap);

    \draw[->, >=stealth] (confidentInCatalog.west) -- ++(-1.1cm, 0cm) node[anchor=south, xshift=0.5cm]{No}
    |- (pickQueryStars.west);
    \draw[->, >=stealth] (confidentAfterFilter.west) -- ++(-1.1cm, 0cm) node[anchor=south, xshift=0.5cm]{No}
    |- (pickQueryStars.west);
    \draw[->, >=stealth](confidentInMap.west) -- ++(-1.1cm, 0cm) node[anchor=south, xshift=0.5cm]{No}
    |- (pickQueryStars.west);
\end{tikzpicture}
    \caption{
    Flowchart depicting a general star identification process.
    Given an image $I$, this process returns a bijection $h$ between some subset of the input $b$ and a subset of the
    catalog $r$.
    In the event all subsets are exhausted, the function $h: b \rightarrow \emptyset$ is returned (not depicted).
    } \label{figure:genericIdentificationMethodFlowchart}
    }
\end{figure}

\section{Star Identification Methods}\label{sec:starIdentificationMethods}
Six different approaches to star identification are described in this section.
The majority of the literature specifying these methods do not include pseudocode, rather they specify descriptions 
of specific processes used by each method.
Each algorithm is composed of these processes, structured to follow a general identification flow.
When the description did not specify a step in this flow, a naive or existing approach from another method was used.

\subsection{Generic Identification Method}\label{subsec:genericIdentificationMethod}
Each identification method is presented with an image $I$ of size $n$ containing all the stars in the image reference,
as well as a catalog of known stars $K$.
All stars in $I$ exist in the body frame $\iFrame$, and all stars in $K$ exist in the inertial frame $\kFrame$.
The goal of each method is to find some bijection between a subset of the image stars $b$ and a subset
of the catalog stars $r$.
This function is denoted as $h$ with domain $b$ and codomain $r$.
Complete identification of the all stars in each image is not the focus.

\begin{subequations}
    Every algorithm starts with some combination from all possible $d$ combinations of $n$ stars $C(n, d)$, where
    $d = $ the size of the image subset that specific identification method uses.
    $b$ is selected using one of these combinations.
    For an identification method that uses $d=2$ stars to determine the mapping in an image of $n=4$ stars, the
    combinations of $I$ are:
    \begin{align}
        C(4, 2) \text{ of } I &= \set{\set{\vv{I_1}, \vv{I_2}}, \set{\vv{I_1}, \vv{I_3}}, \ldots,
        \set{\vv{I_3}, \vv{I_4}}} \\
        C(4, 2) \text{ of } I &= \set{ b_1, b_2, \ldots, b_6 }
    \end{align}
\end{subequations}

\begin{subequations}
    There exists a set $K^d$, composed of $d$ sized sets of all possible combinations (or permutations) of stars
    from the catalog $C$.
    Using certain features of the image star subset, the entire $K^d$ set is filtered to a set of catalog star
    candidates $R$.
    This is known as the $K^d$ filtering step, or the catalog query step.
    Referencing the same $d=2$ identification method as before, an image subset $b = \set{\vv{I_1}, \vv{I_2}}$ may
    yield the candidates in~\autoref{eq:catalogCandidateExample}:
    \begin{align}
        \label{eq:catalogCandidateExample}
        R &= \set{ \set{\vv{K_{104}}, \vv{K_{899}}}, \set{\vv{K_{7622}}, \vv{K_{7771}}}, \ldots) } \\
        R &= \set{ r_1, r_2, \ldots }
    \end{align}
\end{subequations}

Through some filter process or restriction criteria for $R$ itself, a single set $r$ from the catalog star candidates is
eventually selected.
This may require going through multiple catalog candidate sets and repeating the catalog query step.
This is known as the $R$ filtering step.
For a process with the $R$ restriction criterion of $|R| = 1$, the following sequence
of events may occurring before finding a single $r$ set.
\begin{align*}
    &t = 1, \text{ query with } b^{(1)}, \text{ get } R^{(1)} = \{\vv{r_{11}}, \vv{r_{12}}, \ldots\}. \\
    &t = 2, \ \abs{ R^{(1)} } \neq 1, \text{ criterion not met. } \\
    &t = 3, \text{ choose new image subset } b^{(2)}. \\
    &t = 4, \text{ query with } b^{(2)}, \text{ get } R^{(2)} = \{\vv{r_{21}}\}. \\
    &t = 5, \ \abs{ R^{(2)} } = 1, \text{ criterion met. } \\
    &t = 6, \text{ return } r, r \in R^{(2)} (\text{sole element in } R^{(2)} ).
\end{align*}

From here, a bijection $h: b \rightarrow r$ is determined that maps each star found in the image star subset to a
single star in the catalog candidate set.
If we are not confident in $h$ at this point, another image star subset is chosen and the process is repeated.
If we are confident in $h$, then $b$, $r$, and $h$ are returned.
This process is detailed in~\autoref{figure:genericIdentificationMethodFlowchart}.
In the event no map can be determined, an error is raised and the function $h: b \rightarrow \emptyset$ is returned
instead.

% Leaving this out for now. The flowchart should explain the process better.
%\begin{algorithm}[ht] \setstretch{1.0}[H]
%    \caption{Generic Star Identification Method} \label{algorithm:genericStarIdentification}
%    \begin{algorithmic}[1]
%        \Procedure{Identify}{}
%        \State $I \gets $ all stars from image
%        \For{$c \in r_k^{|I|}$}
%        \State $b \gets \{I(c_1), I(c_2), I(c_3), \dots, I(r_k)\}$
%        \State $R \gets $ catalog star sets, each set of size $=k$
%        \State $r \gets $ a single set from $R$ that matches $b$
%        \State $a \gets $ a map between $b$ and $r$
%        \\
%        \If{the steps above are successful}
%        \State \textbf{return} $a$
%        \EndIf
%        \EndFor
%        \EndProcedure
%    \end{algorithmic}
%\end{algorithm}

% Citation: https://digitalcommons.usu.edu/cgi/viewcontent.cgi?article=2723&context=etd AND Gottlieb.
\subsection{Angle Method}\label{subsec:angleMethod}
In 1978, Gottlieb developed the Polygon Angular Matching method~\cite{Angle,AnalysisUncompensated}.
This method has been adjusted to fit the general flow posed in~\autoref{subsec:genericIdentificationMethod}.

Given a set of stars from the image $I$, $d = 2$ stars are selected to obtain the $b$ set.
The original literature does not specify how to choose the image subset, so the naive approach is taken.
The selection order is governed by lines (2) and (3) in~\autoref{algorithm:angleIdentification}.
This fixes the star $\vv{b_1}$ in $b$ for $n$ image star subset selections, while constantly changing
$\vv{b_2}$ for every new $b$ choice.
An example sequence of pairs is depicted below for $n = 3$ stars.
\begin{equation}
    C(3, 2) \text{ of } I = \left(\set{\vv{I_1}, \vv{I_2}}, \set{\vv{I_1},\vv{I_3}}, \set{\vv{I_2},\vv{I_1}},
    \ldots \right)
\end{equation}

The catalog query step searches for pairs in the $K^2$ catalog whose angular separation is close to the angular
separation of the image star subset.
For the image star subset, the origin of the angular separation calculation $\theta(b)$ is the focal point of the lens
itself.
For a catalog star candidate set, the origin of this calculation $\theta(r)$ is the center of the Earth.
To obtain $R$, the predicate $P_\theta(r, b)$ is used to filter the $K^2$ catalog:
\begin{equation}\label{eq:angleRequirement}
    P_{\theta}(r, b) : \left\lvert \theta(r) - \theta(b)\right\rvert < 3 \sigma_\theta
\end{equation}
where $\sigma_{\theta}$ represents the deviation of the uncertainty between the $\theta$ computation from star sensor
measurements and the same $\theta$ computation from stars defined in the catalog.
Assuming the noise follows a Gaussian distribution, it follows that 99.7\% of all true pairs will be within this range
~\cite{Spherical}.

Once the catalog candidates are obtained, the $\abs{R} = 1$ criterion is imposed, repeating this process until only 
one candidate exists in $R$.
This sole element $\vv{R_1}$ is then selected to be $r$.

\begin{subequations}
    To determine the most likely bijection $h$, Needlemen specifies the use of a \textit{direct match test}
    (DMT)~\cite{Tappe,DMT}.
    Given an image star pair $b$ and a catalog star pair $r$ for, the following is proposed:
    \begin{equation}
    h_1 : h_1\left(\vv{b_1}\right) \rightarrow \vv{r_1}, h_1\left(\vv{b_2}\right) \rightarrow \vv{r_2}
    \end{equation}
    Wahba's problem is then solved using the TRIAD method to obtain a rotation $A_1$ between the image and catalog
    frames.
    This process is repeated for the other possible permutation to obtain a second rotation $A_2$:
    \begin{equation}
        h_2 : h_2\left(\vv{b_1}\right) \rightarrow \vv{r_2}, h_2\left(\vv{b_2}\right) \rightarrow \vv{r_1}
    \end{equation}
\end{subequations}
The most likely attitude is determined by the \Call{FPO}{} method, which returns how many stars from $I$ align with
$K$ given $A_1$ or $A_2$.
The bijection with the most stars is then returned.
If all bijections return sets with the same size, then we are not confident in any of our choices and
return the function $h: \vv{b} \rightarrow \emptyset $.

%The \Call{FPO}{} method determines the number of stars from $I$ in the body frame $\iFrame$ align with $C$ in the
%inertial frame $\kFrame$.
%More stars are selected from the catalog that are near the original subset $r$, denoted as $P$.
%The \Call{FPO}{} method then filters out stars in set $P$ that do not overlay with some star in $I'$, the image set
%rotated by some rotation $q$.
%An additional term $\sigma_o$ is defined, which defines how selective this filter.
%$\sigma_o$ can also be thought of as an assumption of the noise associated with the rotation $q$.

\newcommand{\invalidBijection}{\If{$\forall \ \vv{b^\star}, \ \vv{b^\star} \in b \land h\left(\vv{b^\star}\right)
\neq \emptyset$}}
\begin{algorithm}
    \caption{Angle Identification Method} \label{algorithm:angleIdentification}
    \begin{algorithmic}[1]
        \Function{FPO}{$P$, $I$, $A$}
        \State $I' \gets$ stars in $I$ rotated by $A$
        \State $\bar{P} \gets $ \{$p \in P \ | \ \exists \ i \ (i \in I' \land \theta (i, p) < 3\sigma_o)$\}
        \State \textbf{return} $\bar{P}$ \Comment Stars in $P$ that \textit{overlay} with $I'$.
        \EndFunction
        \\
        \Function{DMT}{$b, r, I$}
        \State $H \gets $ all possible bijections of $b$ and $r$
        \State $P \gets $ all stars in catalog near $r$, $M \gets \emptyset$
        \For {$h \in H$}
        \State $A \gets $ \Call{TRIAD}{$h, b, r$}, $M_h \gets $ \Call{FPO}{$P, I, A$}
        \EndFor
        \If{$|M_0| = |M_1| = \ldots = \abs{M_\abs{M}}$}
        \State \textbf{return} $h : b \rightarrow \emptyset $ \Comment Not confident in result.
        \Else
        \State \textbf{return} $h \in H$ associated with largest set in $M$
        \EndIf
        \EndFunction
        \\
        \Function{Identify}{$I, K^2$}
        \For{$i \gets 1 \text{\textbf{ to }} n$} \Comment Iterate through $C(n, 2)$.
        \For{$j \gets i + 1 \text{\textbf{ to }} n - 1$}
        \State $b \gets \left(\vv{b_i}, \vv{b_j}\right)$, $R \gets \{ r \mid r \in K^2 \land P_\theta(r, b) \}$
%        \State $R \gets $ catalog pairs that meet~\algeqref{eq:angleRequirement} with $b$
        \If{$\lvert R \rvert = 1$}
        \State $h \gets $ \Call{DMT}{$b, R_1, I$} % TODO: Maybe change this notation...
        \invalidBijection
        \State \textbf{return} $b, r, h$
        \EndIf
        \EndIf
        \EndFor
        \EndFor
        \EndFunction
    \end{algorithmic}
\end{algorithm}

Accessing the catalog is the most expensive operation for all of the identification methods.
Consequently, the running time of this algorithm $T_{angle}$ can be described in terms of the number of queries and
the number of entries that exist in the $K^2$ catalog.
There exist $2n^2$ catalog accesses at worst, requiring two catalog accesses (query step and \Call{DMT}{} calls) for
each combination of pairs in $I$.
The second term describes the number of comparisons until $r$ sets are found and are able to be returned.
Given a B+ tree indexed database with $\abs{K^2} = m_2$ elements, there exists no more than 
$\bigO \left( \log(m_2) \right)$ comparisons.
%\begin{equation}\label{eq:complexityAnglePart1}
%    T_{angle} = \bigO\left(C(n, 2) \cdot n \cdot \log(m_{angle}) \right)
%\end{equation}
%We expand the combination term to get:
%\begin{equation}
%    C(n, 2) = \frac{n(n - 1)}{2} = \frac{n^2}{4} - \frac{n}{4}
%\end{equation}
%~\autoref{eq:complexityAnglePart1} then simplifies to~\autoref{eq:complexityAngle}:
\begin{equation}\label{eq:complexityAngle}
    T_{angle} = \bigO\left( n^2 \cdot \log(m_2) \right)
\end{equation}

All methods associated with the Angle method are written in~\autoref{algorithm:angleIdentification}.

\subsection{Dot Angle Method}\label{subsec:dotAngleMethod}
In 1995, Liebe developed the Liebe Star ID method~\cite{DotAngle,AnalysisUncompensated}.
This method has been adjusted to fit the general flow posed in~\autoref{subsec:genericIdentificationMethod}.

Starting with an image $I$, a central star $\vv{b_c}$ is selected.
A new central star selection does not involve generating permutations like the Angle method, rather it involves
iterating through $I$ in a sequential manner.
The two closest stars in the image to the central star are selected next, denoted as $\vv{b_{c1}}$ and $\vv{b_{c2}}$.

The catalog query step searches for trios in the $\bar{K^3}$ catalog whose features are close to the same features of
the image subset.
Unlike the Angle method's $K^d$ set, $\bar{K^d}$ is defined to be all \textit{permutations} of size $d$ rather than
combinations.
These features are defined as the angular separation between the first closest star and the central star
($\theta\left(\vv{b_{c1}}, \vv{b_c}\right)$ vs. $ \theta\left(\vv{r_{c1}}, \vv{r_c}\right)$), the
angular separation between the second closest star and the central star, ($\theta\left(\vv{b_{c2}},
\vv{b_c}\right) $ vs. $\theta\left(\vv{r_{c2}}, \vv{r_c}\right)$),
and the angular separation between the two closest stars with the central star as the origin instead of the Earth or 
focal point ($\phi(b) $ vs. $ \phi(r)$).
To obtain $R$, the predicate $P_{\theta, \phi}(r, b)$ is used to filter the $\bar{K^3}$ catalog:
\begin{equation}
    \begin{aligned}
        P_{\theta, \phi} (r, b): \left\lvert \theta(\vv{r_{c1}}, \vv{r_c}) - \theta(\vv{b_{c1}}, \vv{b_c})\right\rvert
        &< 3 \sigma_{\theta} \ \land \\ \left\lvert \theta(\vv{r_{c2}}, \vv{r_c}) - \theta(\vv{b_{c2}}, 
        \vv{b_c})\right\rvert &< 3 \sigma_{\theta} \ \land \\ \left\lvert \phi(r) - \phi(b)\right\rvert &< 3 
        \sigma_\phi \ \land \\\theta(\vv{r_{c1}}, \vv{r_c}) &< \theta(\vv{r_{c2}}, \vv{r_c})
    \end{aligned}
\end{equation}
where $\sigma_{\theta}$ and $\sigma_{\phi}$ represent the deviation of measurement to catalog uncertainty of the
$\theta$ and $\phi$ computations respectively.

The original literature states that this should be repeated for all $b$ sets.
Given that complete identification of $I$ is not the goal, this method has been adjusted to query with one $b$ at a
time.
After finding some $R$ that meets the same $R$ criterion as the Angle method, the bijection:
\begin{equation}
    h: h\left(\vv{b_1}\right) \rightarrow \vv{r_1}, h\left(\vv{b_2}\right) \rightarrow \vv{r_2}, h\left(\vv{b_3}\right)
    \rightarrow \vv{r_3}
\end{equation}
is constructed and returned.
By imposing the last term in predicate $P_{\theta, \phi}(r, b)$ at query time (last term in) and by searching all
permutations instead of combinations, the need for a star mapping procedure (e.g.\ direct match test) is
no longer required.
Storing permutations does increase the storage required for the $\bar{K^3}$ catalog though, which begs the question,
"Does this extra space aid in accuracy or runtime?".

%The star trios in the catalog star candidates represent potential catalog maps for the image star trio $(b_i, b_j,
%b_c)$.
%Liebe's original method states that this process should be repeated for all stars in the image, meaning that all
%stars will be the central star at one point
%By the end, each star in the image will have accrued a set of possible catalog matches $Y$.
%The complete $I \rightarrow R$ map is found by picking the most frequent catalog star appearing in each $Y$
%set~\cite{DotAngle,AnalysisUncompensated}.
%
%To more closely follow the generic star identification flow, we modified the Dot Angle method to not store $Y$,
%requiring that only one central star choice is needed to acquire a total match.
%If a confident match is not found by the first central star, then the search process will be repeated until such a
%match is found.

\begin{algorithm}
    \caption{Dot Angle Identification Method} \label{algorithm:dotAngleIdentification}
    \begin{algorithmic}[1]
        \Function{Identify}{$I, \bar{K^3}$}
        \For{$c \gets 1 \text{\textbf{ to }} n$}  \Comment Iterate through all of $I$.
        \State $\theta_I \gets \{\theta(\vv{b_c}, \vv{b_i}) \mid \vv{b_i} \in I\}$ \Comment $\theta$ 
        (all stars, $\vv{b_c}$).
        \State $\vv{b_{c1}} \gets \vv{b_i}$ associated with smallest $\theta$ in $\theta_I$
        \State $\vv{b_{c2}} \gets \vv{b_i}$ associated with 2nd smallest $\theta$ in $\theta_I$
        \State $b \gets \left(\vv{b_c}, \vv{b_{c1}}, \vv{b_{c2}}\right)$, $R \gets \set{ r \mid r \in \bar{K^3} \land
        P_{\theta, \phi}(r, b) }$
        \If{$\lvert R \rvert = 1$}
        \State \textbf{return} $b, r, h: h\left(\vv{b_c}\right) \rightarrow \vv{r_c}, h\left(\vv{b_{c1}}\right)
        \rightarrow \vv{r_{c1}},$
        \State \ \ \ \ \ \ \ \ \ \ \ \ \ \  \ \ \ \ \ \ $h\left(\vv{b_{c2}}\right) \rightarrow \vv{r_{c2}}$
        \EndIf
        \EndFor
        \EndFunction
    \end{algorithmic}
\end{algorithm}

The running time of this algorithm $T_{dot}$ is depicted below, again described in terms of the number of queries
and $\bar{K^3}$ catalog entries:
\begin{equation}\label{eq:dotComplexity}
    T_{dot} = O\left( n \cdot \log(\bar{m_3}) \right)
\end{equation}
where $\bar{m_3}$ is the size of the $\bar{K^3}$ catalog.

This entire method is described in~\autoref{algorithm:dotAngleIdentification}.

\subsection{Spherical Triangle Method}\label{subsec:sphericalTriangleMethod}
In 2004, Cole and Crassidus developed the Spherical Triangle method~\cite{Spherical}.
This method has been adjusted to fit the general flow posed in~\autoref{subsec:genericIdentificationMethod}.

Given a set of stars from the image $I$, $d = 3$ stars are selected to obtain the $b$ set.
Again, the $b$ selection is not specified here so we assume the naive approach with $C(n, 3)$ combinations.
The star $\vv{b_1}$ is fixed in $b$ for $n^2$ image star subset selections, the star $\vv{b_2}$ is fixed for
$n$ selections, and the last star $\vv{b_3}$ is constantly changed for every new $b$ choice.

The catalog query step searches for trios in the $K^3$ catalog whose spherical area and moment
are close to the spherical area and moment of the image star subset.
For the image star subset, the spherical area and moment are represented as $a(b)$ and $\tau(b)$ respectively.
For the catalog star candidate set, these same features are represented as $a(r)$ and $\tau(r)$.
To obtain $R$, the predicate $P_{a, \tau}(r, b)$ is used to filter the $K^3$ catalog:
\begin{equation}
    \begin{aligned}
        P_{a, \tau}(r, b) : \abs{ a(r) - a(b)} &< 3\sigma_{a}
        \ \land \\ \abs{\tau(r) - \tau(b)} &< 3\sigma_{\tau}
    \end{aligned}
\end{equation}
where $\sigma_a$ and $\sigma_\tau$ represent the deviation of measurement to catalog uncertainty of spherical area
and moment computations respectively.

Unlike the previous two methods, the $R$ criterion of $|R| = 1$ not being met does not lead to an immediate new
selection of $b$.
Instead, the candidate set itself is reduced by \textit{pivoting} until the criterion is met or pivots can no longer
be performed.
The procedure starts by querying the catalog again for a second set of catalog candidate sets $\bar{R}$ with a
different image star subset $\bar{b} = \left(\vv{b_i}, \vv{b_j}, \vv{\beta}\right)$.
In $\bar{b}$, the first two stars are held constant while the third star is swapped with another in $I$ that was not
already used in this specific pivot.
All star trios in the initial search that do not match a trio in the second search by \textit{two stars} (a partial
match) are removed from the initial search candidate star set.
A pivot uses at most $n - 3$ additional catalog accesses, but prevents wasting a catalog candidate set that may contain
the correct $r$ set for the given $b$.

\begin{subequations}
    Cole and Crassidus do not specify identification steps, so the DMT process is used to complete the star
    identification process.
    Given an image star trio and a catalog star trio, an bijection is proposed:
    \begin{equation}
        h_1 : h_1\left(\vv{b_1}\right) \rightarrow \vv{r_1}, h_1\left(\vv{b_2}\right) \rightarrow \vv{r_2},
        h_1\left(\vv{b_{3}}\right) \rightarrow \vv{r_3}
    \end{equation} \label{eq:trianglePossibleMaps}
    The TRIAD method only uses two vector observations from each frame, meaning that the $\vv{b_3} \rightarrow \vv{r_3}$
    pairing is disregarded as the first rotation $A_1$ is computed.
    This process is repeated for all 5 other possible bijections to get $A_2, A_3, \dots, A_6$.
    \begin{align}
        h_2 &: h_2\left(\vv{b_1}\right) \rightarrow \vv{r_1}, h_2\left(\vv{b_2}\right) \rightarrow \vv{r_3},
        h_2\left(\vv{b_3}\right) \rightarrow \vv{r_2} \\
        h_3 &: h_3\left(\vv{b_1}\right) \rightarrow \vv{r_2}, h_3\left(\vv{b_2}\right) \rightarrow \vv{r_1},
        h_3\left(\vv{b_3}\right) \rightarrow \vv{r_3} \\
        h_4 &: h_4\left(\vv{b_1}\right) \rightarrow \vv{r_2}, h_4\left(\vv{b_2}\right) \rightarrow \vv{r_3},
        h_4\left(\vv{b_3}\right) \rightarrow \vv{r_1} \\
        h_5 &: h_5\left(\vv{b_1}\right) \rightarrow \vv{r_3}, h_5\left(\vv{b_2}\right) \rightarrow \vv{r_1},
        h_5\left(\vv{b_3}\right) \rightarrow \vv{r_2} \\
        f_{6} &: h_6\left(\vv{b_1}\right) \rightarrow r_4, h_6\left(\vv{b_2}\right) \rightarrow \vv{r_2},
        h_6\left(\vv{b_3}\right) \rightarrow \vv{r_1}
    \end{align}
\end{subequations}
For all six attitudes, the bijection yielding the most aligned stars is returned.

\begin{algorithm}[ht]
    \caption{Triangle Method Identification} \label{algorithm:triangleIdentification}
    \begin{algorithmic}[1]
        \Function{PartialMatch}{$R, \bar{R}$}
        \ForAll {$\bar{r} \in \bar{R}$}
        \LineComment $\bar{r}$ and $r$ share two stars.
        \If {$\exists \ r \ | \ (r \in R \land |r \cap \bar{r}| = 2)$}
        \State $R_{new} \gets \bar{R} \cup \{\bar{r}\}$
        \EndIf
        \EndFor
        \State \textbf{return} $R_{new}$
        \EndFunction
        \\
        \Function{Pivot}{$\vv{b_i}, \vv{b_j}, \vv{b_k}, R$}
        \State $b \gets (\vv{b_j}, \vv{b_j}, \vv{b_k})$, $\bar{R} \gets \set{ \bar{r} \mid \bar{r} \in K^3
        \land P_{a, \tau}(\bar{r}, b) }$
        \State $R' \gets $ \Call{PartialMatch}{$R, \bar{R}$}
        \If{$\abs{R'} = 1 \lor \abs{R'} = 0 $}
        \State \textbf{return} $R'$ \Comment $R'$ is either $\emptyset$ or a single $r$.
        \Else
        \State $\vv{\beta} \gets \text{an unused star in this pivot}$
        \State \textbf{return} \Call{Pivot}{$\vv{b_i}, \vv{b_j}, \vv{\beta}, R'$}
        \EndIf
        \EndFunction
        \\
        \Function{Identify}{$I, K^3$}
        \For{$i \gets 1 \text{\textbf{ to }} n$}  \Comment Iterate through $C(n, 3)$.
        \For{$j \gets i + 1 \text{\textbf{ to }} n - 1$}
        \For{$k \gets j + 1\text{\textbf{ to }} n - 2$}
        \State $b \gets \left(\vv{b_i}, \vv{b_j}, \vv{b_k}\right)$, $R \gets \set{ r \mid r \in K^3
        \land P{a, \tau}(r, b) }$
        \If{$|R| \neq 1$} \Comment Pivot if necessary.
        \State $R \gets $ \Call{Pivot}{$\vv{b_i}, \vv{b_j}, \vv{b_k}, R$}
        \EndIf
        \If{$R \neq \emptyset$} \Comment Verify the pivot's success.
        \State $h \gets $ \Call{DMT}{$b, R_1, I$}
        \invalidBijection
        \State \textbf{return} $b, R_1, h$
        \EndIf
        \EndIf
        \EndFor
        \EndFor
        \EndFor
        \EndFunction
    \end{algorithmic}
\end{algorithm}

The running time of this algorithm $T_{sphere}$ is depicted below in terms of the number of queries and the number of
entries in the $K^3$ catalog.
At most, this requires $2n^4$ catalog access: $n^3$ for each combination of trios in $I$, $n - 3$ potential catalog
accesses incurred for each pivot, and an additional $n^4$ queries with each \Call{DMT}{} call.
\begin{equation}\label{eq:sphereComplexity}
    T_{sphere} = \bigO\left( n^4 \cdot \log (m_3) \right)
\end{equation}
where $m_3$ represents the size of the $K^3$ catalog.

This entire method is described in~\autoref{algorithm:triangleIdentification}.

\subsection{Planar Triangle Method}\label{subsec:coleAndCrassidus'sPlanarTriangleMethod}
In 2006, Cole and Crassidus developed the Planar Triangle method~\cite{Planar}.
This method has been adjusted to fit the general flow posed in~\autoref{subsec:genericIdentificationMethod}.

This is identical to their Spherical Triangle method, with the exception that each image trio is represented as a
planar triangle instead of a spherical one.
This results in the computation of a planar area and moment as opposed to a spherical area and moment.

% Mortari introduced the use of \textit{search-less} catalog access using the $k$-vector approach, but this will not be
% discussed in this paper.
\subsection{Pyramid Method}\label{subsec:pyramidMethod}
In 2004, Mortari developed the Pyramid method~\cite{Pyramid,AnalysisUncompensated}.
This method has been adjusted to fit the general flow posed in~\autoref{subsec:genericIdentificationMethod}.

Given a set of stars from the image $I$, $d = 3$ stars are selected to obtain the $b$ set.
The selection order is governed by lines (27), (28), and (29) in~\autoref{algorithm:pyramidIdentification}.
As opposed to the selection order of the Angle and triangle methods, the $\vv{b_1}$ star in $b$ is no longer fixed
for $n$ or $n^2$ image star subset selections.
This is meant to avoid the persistence of badly recorded or false stars for more than a few combinations.
An example sequence of trios is depicted below for $n = 5$ stars.
\begin{equation}
    \begin{aligned}
        C(5, 3) \text{ of } I = ( &\set{\vv{I_1}, \vv{I_2}, \vv{I_3}}, \set{\vv{I_2}, \vv{I_3}, \vv{I_4}}, \\
        &\set{\vv{I_3}, \vv{I_4}, \vv{I_5}}, \set{\vv{I_1}, \vv{I_2}, \vv{I_4} }\ldots )
    \end{aligned}
\end{equation}

The literature states that a unique and identifiable trio is to be found after the query step using $\theta$,
but does not specify how nor is source code provided.
One approach starts by querying the following items with $b$: pairs from the $K^2$ catalog whose angular separation
is close to the angular separation of $\set{\vv{b_i}, \vv{b_j}}$, pairs from the catalog whose angular separation
is close to the angular separation of $\set{\vv{b_i}, \vv{b_k}}$, and pairs from the catalog whose angular separation is
close to the angular separation of $\set{ b_j, \vv{b_k} }$.
Each of these sets are denoted as $T_{ij}, T_{ik}$ and $T_{jk}$ respectively.
These sets are then flattened from sets of pairs to just a single set of stars (\Call{FC}{}
in~\autoref{algorithm:pyramidIdentification}) and the difference of two flattened sets identify candidates for that
star.
For the sets of pairs $T_{ij}, T_{jk}$ found by querying with $P_\theta$ and $\set{\vv{b_i}, \vv{b_j}},
\set{\vv{b_j}, \vv{b_k}}$, the common star between each $b$ set is $\vv{b_j}$.
An example of finding catalog candidates for $b_j$ with this method is given below:
\begin{equation}
    \begin{aligned}
        T_{ij} \gets& \set{ \set{\vv{K_{1123}}, \vv{K_{9001}}}, \set{\vv{K_{8234}}, \vv{K_{33}}} } \\
        T_{jk} \gets& \set{ \set{\vv{K_{612}}, \vv{K_{1123}}}, \set{\vv{K_{33}}, \vv{K_{345}}} } \\
        T_j =& \  \Call{FC}{T_{ij}, T_{jk}} = \set{\vv{K_{1123}}, \vv{K_{33}}}
    \end{aligned}
\end{equation}
$R$ is found by repeating the process above for $T_i$ and $T_k$, and generating all possible sequences.
This is depicted in the \Call{Query}{} function in~\autoref{algorithm:pyramidIdentification}.

After finding some $R$ where $\abs{T_i} = \abs{T_j} = \abs{T_k} = 1$ (same criterion as Angle and Dot Angle), a
verification step is performed.
A different star from the image $\vv{\beta}$ is selected and the query step is performed for each distinct combination
of $b$ and $\vv{\beta}$.
If $\abs{T_\beta} \neq 1$, then verification step has failed and another image subset is selected.
Otherwise, the bijection $h : h\left(\vv{b_1}\right) \rightarrow \vv{r_1}, h\left(\vv{b_2}\right)
\rightarrow \vv{r_2}, h\left(\vv{b_3}\right) \rightarrow(\vv{b_3})$ is returned.
Like the Dot Angle method, a star mapping procedure is not required because each individual star is identified
at query time.

\begin{algorithm}[ht]
    \caption{Pyramid Identification Method} \label{algorithm:pyramidIdentification}
    \begin{algorithmic}[1]
        \Function{FC}{$T_1, T_2$}
        \LineComment Flatten $T_1, T_2$ from set of sets to a set.
        \State $\bar{T_1} \gets \emptyset, \bar{T_2} \gets \emptyset$
        \ForAll{$i \in \set{1, 2}$}
        \ForAll{$\vv{t} \in T_i$}
        \State $\bar{T_i} \gets \bar{T_i} \cup \set{ \vv{t_1}, \vv{t_2} }$
        \EndFor
        \EndFor
        \State \textbf{return} $\bar{T_1} \cap \bar{T_2}$
        \EndFunction
        \\
        \Function{FindT}{$\vv{b^1}, \vv{b^2}, \vv{b^3}, K^2$}
        \LineComment $\abs{\vv{b^1}} = \abs{\vv{b^2}} = \abs{\vv{b^3}} = 1$, search with each pair.
        \State $T_1 \gets \set{ r \mid r \in K^2 \land P_\theta \left(r, b^1\right) }$
        \State $T_2 \gets \set{ r \mid r \in K^2 \land P_\theta \left(r, b^2\right) }$
        \State $T_3 \gets \set{ r \mid r \in K^2 \land P_\theta \left(r, b^3\right) }$
        \State \textbf{return}{$\left( T_1, T_2, T_3 \right)$}
        \EndFunction
        \\
        \Function {Query}{$b, K^2$}
        \State $(T_{ij}, T_{ik}, T_{jk}) \gets$ \Call{FindT}{$\set{ \vv{b_i}, \vv{b_j} }, \set{ \vv{b_i}, \vv{b_k} }, $
        \State \ \ \ \ \ \ \ \ \ \ \ \ \ \ \ \ \ \ \ \ \ \ \ \ \ \ \ \ \ \ \ \
        $\set{ \vv{b_j}, \vv{b_k} }, K^2$}, $R \gets \emptyset$
        \ForAll{$t_i \in$ \Call{FC}{$T_{ij}, T_{ik}$}}
        \ForAll{$t_j \in$ \Call{FC}{$T_{ij}, T_{jk}$}}
        \ForAll{$t_k \in$ \Call{FC}{$T_{ik}, T_{jk}$}}
        \State $R \gets R \cup \set{ \left(\vv{t_i}, \vv{t_j}, \vv{t_k}\right) }$
        \EndFor
        \EndFor
        \EndFor
        \State \textbf{return} $R$ \Comment Return all permutations from $T$ sets.
        \EndFunction
        \\
        \Function{Identify}{$I, K^2$}
        \LineComment Iterate through $C(n, 3)$ while avoiding false stars.
        \For{$dj \gets 1 \text{\textbf{ to }} n - 2$}
        \For{$dk \gets 1 \text{\textbf{ to }} n - 1 - dj$}
        \For{$i \gets 1 \text{\textbf{ to }} n - dj - dk$}
        \State $j \gets i + dj$, $k \gets j + dk$
        \State $b \gets left(\vv{b_i}, \vv{b_j}, \vv{b_k}\right)$, $R \gets $\Call{Query}{$b, K^2$}
        \If{$\abs{R} = 1$}
        \LineComment Verification step below.
        \State $\vv{\beta} \gets $ single star in $I$ where $\vv{\beta} \notin b$
        \State $(T_{ij}, T_{ik}, T_{jk}) \gets$ \Call{FindT}{$\set{ \vv{b_i}, \vv{\beta} }, \set{ \vv{b_j},
        \vv{\beta} },$
        \State \ \ \ \ \ \ \ \ \ \ \ \ \ \ \ \ \ \ \ \ \ \ \ \  \ \ \ \ \ \ \ \ $\set{ \vv{b_k}, \vv{\beta} }, K^2$}
        \State $T_\beta \gets $ \Call{FC}{$T_{i\beta}, T_{j\beta}$} $\cap$ \Call{FC}{$T_{j\beta}, T_{k\beta}$}
        \If{ $\abs{T_\beta} = 1$}
        \State \textbf{return} $b, r, h: h\left(\vv{b_1}\right) \rightarrow \vv{r_1}, h\left(\vv{b_2}\right)
        \rightarrow \vv{r_2},$
        \State \ \ \ \ \ \ \ \ \ \ \ \ \ \ \ \ \ \ \ \ $h\left(\vv{b_3}\right) \rightarrow \vv{r_3}$
        \EndIf
        \EndIf
        \EndFor
        \EndFor
        \EndFor
        \EndFunction
    \end{algorithmic}
\end{algorithm}

The running time of this algorithm $T_{pyramid}$ is depicted below in terms of the number of queries and the number
of entries in the $K^2$ catalog.
At most, this requires $6n^3$ catalog accesses: $3n^3$ accesses for each query step with an additional $3n^3$ accesses
for each verification step.
\begin{equation}
    T_{pyramid} = \bigO \left( n^3 \cdot \log( m_2 ) \right)
\end{equation}
where $m_2$ is the size of the $K^2$ catalog.

This entire method is specified in~\autoref{algorithm:pyramidIdentification}.

\subsection{Composite Pyramid Method}\label{subsec:compositePyramidMethod}
In 2014, Toloei developed the Novel Stars ID method, which retains all of the key aspects of Motari's
Pyramid method but uses the features from Cole and Crassidus's Planar Triangle method for the query and reference steps
instead of angles~\cite{Composite}.
This method has been adjusted to fit the general flow posed in~\autoref{subsec:genericIdentificationMethod}.

Given a set of stars from the image $I$, $d = 3$ stars are selected in same manner as the Pyramid method to obtain
the $b$ set.
From here, the process to obtain the $R$ set is the same as the triangle methods: use $P_{a, \tau}(r, b)$ and $b$ to
select all candidates from $K^3$.
If the current $R$ set meets the same $\abs{R} = 1$ criterion, then a similar verification step to the Pyramid method
is performed with the Planar Triangle features.
Once this test has passed, the \Call{DMT}{} method is used to construct the bijection $h$ to potentially return.
The Pyramid method did not need this call as an implicit bijection was formed through its query process.

\begin{algorithm}[ht]
    \caption{Composite Pyramid Identification Method}\label{algorithm:compositePyramid}
    \begin{algorithmic}[1]
        \Function{Identify}{$I, K^3$}
        \LineComment Iterate through $C(n, 3)$ while avoiding false stars.
        \For{$dj \gets 1 \text{\textbf{ to }} n - 2$}
        \For{$dk \gets 1 \text{\textbf{ to }} n - 1 - dj$}
        \For{$i \gets 1 \text{\textbf{ to }} n - dj - dk$}
        \State $j \gets i + dj$, $k \gets j + dk$
        \State $b \gets (\vv{b_i}, \vv{b_j}, \vv{b_k})$, $R \gets \set{ r \mid r \in K^3 \land P{a, \tau}(r, b) }$
        \If{$\lvert R \rvert = 1$}
        \LineComment Verification step below.
        \State $\beta \gets $ single star in $I$ where $\beta \notin b$
        \State $T_{12\beta} \gets \set{ r \mid r \in K^3 \land P_{a, \tau}(r, \set{\vv{b_1}, \vv{b_2}, \vv{\beta}})}$
        \State $T_{13\beta} \gets \set{ r \mid r \in K^3 \land P_{a, \tau}(r, \set{\vv{b_1}, \vv{b_3}, \vv{\beta}})}$
        \State $T_{23\beta} \gets \set{ r \mid r \in K^3 \land P_{a, \tau} (r, \set{\vv{b_2}, \vv{b_3}, \vv{\beta}})}$
        \State $T_\beta \gets $ \Call{FC}{$T_{12\beta}, T_{13\beta}$} $\cap$ \Call{FC}{$T_{j13\beta}, T_{23\beta}$}
        \If{$\abs{T_\beta} = 1$}
        \State $h \gets$ \Call{DMT}{$b, R_1, I$}
        \invalidBijection
        \State \textbf{return} $h$
        \EndIf
        \EndIf
        \EndIf
        \EndFor
        \EndFor
        \EndFor
        \EndFunction
    \end{algorithmic}
\end{algorithm}

The running time of this algorithm $T_{composite}$ is depicted below in terms of number of queries and the number of}
items in the $K^3$ catalog.
At most, this requires $5n^3$ catalog accesses: $n^3$ for each query step, an additional $3n^3$ accesses for each
verification step, and an additional $n^3$ accesses for each \Call{DMT}{} call.
\begin{equation}
    T_{composite} = \bigO (n^3 \cdot \log(m_3))
\end{equation}
where $m_3$ represents the number of entries in the $K^3$ catalog.

This entire method is specified in~\autoref{algorithm:compositePyramid}.

%\begin{table*}[ht]
%    \centering{
%    \caption{
%    Overview table of the different identification methods.
%    Each method's image features, reduction process, identification process is displayed.
%    } \label{tab:identificationMethodOverview}
%    \begin{tabular}{  m{0.22\linewidth} || m{0.21\linewidth} | m{0.21\linewidth} | m{0.24\linewidth} }
    & \textbf{Image Features} & \textbf{Reduction} & \textbf{Identification} \\
    \hline \hline
    \textbf{Angle} & $\theta^{ij}$ & Require $\lvert R \rvert=1$ & \Call{DMT}{$b, r, I$} \\ \hline
    \textbf{Dot Angle} & $\theta^{ic}, \theta^{jc}, \phi$ & Require $\lvert R \rvert = 1$ & Restrict $\theta^{ic},
    \theta^{jc}$ at Query Time \\ \hline
    \textbf{Spherical Triangle} & Spherical $a^{ijk}, \imath^{ijk}$ & \Call{Pivot}{$b_i, b_j, b_k, R_1$} &
    \Call{DMT}{$b, r, I$} \\ \hline
    \textbf{Planar Triangle} & Planar $a^{ijk}, \imath^{ijk}$ & \Call{Pivot}{$b_i, b_j, b_k, R_1$} &
    \Call{DMT}{$b, r, I$} \\ \hline
    \textbf{Pyramid} & $\theta^{ij}, \theta^{ik}, \theta^{jk}$ & Require $\lvert R \rvert = 1$ &
    \Call{Common}{$R^{ab}, R^{ac}, F$}, \newline \Call{PyramidVerify}{$r, b, I$} \\ \hline
    \textbf{Composite Pyramid} & Planar $a^{ijk}, \imath^{ijk}$ & Require $\lvert R \rvert = 1$ & \Call{DMT}{$b, r, I$},
    \newline \Call{CompositeVerify}{$r, b, a, I$}
\end{tabular}

%    }
%\end{table*}