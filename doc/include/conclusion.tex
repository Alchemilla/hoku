\section{Conclusion}\label{sec:conclusion}
Attitude determination is a vital part of all spacecraft missions.
Previously, devices such as magnetometers and Sun sensors would give one observation each in both the body and inertial
frames.
The advantage of star trackers is the presentation of multiple observations with just one device.
Star identification is the process associated that maps observations found in the body frame (i.e.\ the image) to the
inertial frame (i.e.\ the catalog).

Gottlieb's Angle method, Liebe's Dot Angle method, Motari's Pyramid method, Cole and Crassidis's Spherical and
Planar Triangle method, and Toloei's Composite Pyramid method were adjusted to fit a general identification flow
and were analyzed in terms of their query, reduction, and identification steps.
Portions that were interchangeable amongst all methods such as database access and centroid determination were
normalized or removed to focus on the star identification aspect itself.

In all experiments, the Pyramid method ranked first (or close to first) in running time and fairly high in accuracy.
The average short running time is a result of a relatively fast query step, and its accuracy can be attributed to the
verification procedure at identification time.
The Pyramid method is the least sensitive to hyperparameter changes in terms of query size response and accuracy
response.
This method produced the lowest number of catalog candidate sets for a given query, and is the most tolerant of
both Gaussian noise and false stars.

The Angle method had the fastest query step, which stems from the small catalog ($\sim$35 times smaller than
the next larger catalogs, the triangle catalogs).
There were a few instances where the accuracy of the Angle method was greater than the Pyramid method, but this came
at the cost of runtime.
The results produced by the query step heavily impacted the reduction and identification runtime.
For the no noise end-to-end case, this method was a factor of MP200 times slower than the next slowest method (Dot
Angle method).

The Dot Angle method is the most sensitive to hyperparameter changes, but had a larger ideal region than the other
methods.
The catalog for this method was $\sim$105 times larger than the Angle method, and consequently had the longest query
step.
This method handles Gaussian noise the best of all other methods, being the second fastest to identify an image and
being first in how accurate the result is.
In terms of false stars, this method ranks last in accuracy.

The accuracy of the Spherical and Planar Triangle methods rank in between for both Gaussian noise and false stars.
The differences between the spherical and planar triangle features are minuscule compared to the algorithms
encapsulating the features themselves.
These methods produced the smallest amount of candidates on average, which was useful in mitigating the highest
theoretical upper bound for catalog accesses.
Unfortunately, these methods were not the fastest nor were they the most accurate.

The Composite Pyramid method, a composite of the Planar Triangle method's features and the Pyramid method's processes,
ranks last in accuracy and average runtime for images with Gaussian noise 2nd in accuracy for images with just
false stars.
Combining the two methods also means combining the filters associated with each, and the results show inconsistent
runtime due to failing at different stages of the algorithm.
Instead of getting the best of both worlds here, the Composite Pyramid method inherits the worst portions of the two.

Overall, the Pyramid method handles both Gaussian noise and false stars the best in a reasonable amount of time.
If one is working with a small field-of-view with limited stars to choose from and Gaussian noise, the Dot Angle
method works best for using one less star than the Pyramid method.
If speed is not a factor but the field-of-view problem still persists, the Angle method should suffice.