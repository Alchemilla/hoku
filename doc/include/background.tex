\section{Introduction}\label{sec:introduction}
With the advent of commercial space industry, we are seeing
an increasing number of spacecrafts, both manned and unmanned,
launched into space. One important function on a spacecraft
is the ability to determine its orientation in space using
the images captured by a camera on the spacecraft - this is
often known as the \textit{Lost-in-space} problem.
For example, consider the design of low Earth
orbit (LEO) spacecrafts. In order for the spacecraft to
point a payload, direct its thrusters, or orient its solar
panels, an accurate \textit{attitude} (another term for
orientation) must be known. There are a few known landmarks
in space where some attitude can be extracted (e.g., the
Earth, the Sun), but this requires constant direction
towards just these objects. Most attitude determination
systems do not limit themselves to using a single object as
a landmark, rather they use multiple stars within the
field of view of their camera to determine their orientation.
The images capture a rectangular region of space -- think of an image of the
sky at night from an earth-centric perspective. Each image is essentially a
collection of
bright spots (celestial objects) and their location in the
image coordinate system. To find the orientation of the
spacecraft, we have to match the pattern of celestial
objects in the image to a database of known celectial
objects. We refer to such database queries as 
\textit{Constellation Queries}: Given a set of query points (in
the image coordinate system), find the sets of points, i.e.,
constellations, in a database of known points (in a standard
coordinate system) that best matches the pattern of the
query points. 

Hypothetically, if we could transform the
query points and database points to a common coordinate
system, then constellation queries would be a simple
database lookup for each query point. Unfortunately,
evaluating constellation queries are complicated by several
issues. First the query points and the points in the
database are in two different coordinate system (also known
as reference frames) and the transformation between the two
coordinate system is not known a priori. In fact, attitude
determination \emph{is} finding the transformation between
the two coordinate system. Second, the query
points are affected by transient celestial objects (e.g.,
meteors) resulting in spurious points in the image,
obstructions resulting in missing points, and camera
charateristics resulting in deviations of the query points
true position.

%Ancient mariners could look up at the night sky, point out which stars they were looking at, and navigate across the
%globe without the use of maps.
%\textit{Star identification algorithms} refer to computational approaches to determining which stars are in the sky.
%Given an image of the sky, star identification is matching the bright spots in an image to stars in an astronomical
%catalog.
%The device that performs these computations is the star tracker, much like the navigators on the ship.
%\textit{Lost-in-space} refers to an additional constraint on the problem: the absence of knowing where we took
%the picture and how we pointed the camera.

%There exist roughly $4{,}500$ stars in the sky visible to the human eye.
%For an image of $n$ stars, the naive approach would be compute $C(4{,}500, n)$ combinations from this collection and
%compare each to some subset of stars found in the image.
%For $n\seq 3$, this requires over $10^{10}$ comparisons.
%As an alternative, we sacrifice storage and precision for speed by searching a separate collection which indexes the
%${\sim}4{,}500$ stars by one or more features.
%When this subset is identified, we determine and return the orientation of the image relative to collection
%of ${\sim}4{,}500$ stars.

%\subsection{Stellar Based Attitude Determination}\label{subsec:stellarBasedAttitudeDetermination}
%Attitude refers to the translation between how one system describes an object compared to how a different system
%describes the same object.
%
%In the context of spacecraft attitude for star identification, there exist three reference frames: the
%\textit{body frame}, the \textit{sensor frame}, and the \textit{inertial frame}.
%The body frame itself is fixed to the structure of the spacecraft, the sensor frame is fixed to the star tracker,
%and the inertial frame refers to some non-accelerating frame in which stellar objects are recorded.
%All observations from the spacecraft exist in the sensor frame, but can easily be rotated to align with the body frame
%(the sensor itself is fixed to the spacecraft chassis).
%Consequently, the body frame is used interchangeably with the sensor frame.
%To describe the craft itself, an inertial frame is required for finding a practical attitude.
%A star observed in the inertial frame is more predictable than the same star observed in a tumbling spacecraft, aiding
%the usage of the attitude with orientation dependent processes.
%Using all three, the goal of attitude determination becomes finding some method of translation between the inertial
%frame and the body frame.

Let $\kFrame$ describe an inertial reference frame for $K$, the objects in the
database, and $\iFrame$ describe a body reference
frame~\cite{wie:spaceVehicleDynamics} for $I$, the set of objects in the query image.
For simplicity, we make the assumption here that all stars in $\kFrame$ are
fixed and exist in an inertial frame known as the \textit{Earth centered
inertial} (ECI) frame. The star vectors themselves come from astronomical
catalogs, recorded as points lying on the celestial
sphere~\cite{tappe:starTrackerDevelopment}. Two pieces of information are given
here: right ascension $\alpha$ (equivalent to latitude on Earth) and
declination $\delta$ (equivalent to longitude). The vector $\vv{K_j}$ represents a point
$\left( \alpha, \delta \right)$ in a 3D spherical frame with a fixed radius for
all stars, projected to 3D Cartesian space.
Let $\vv{I_j}$ represent a 3D point projected from a 2D observation taken by the star tracker.
A basic star tracker is composed of a camera, a computer for determining
orientation, and a link back to the main computer. After taking the picture,
the pixel positions of potential stars in the image are determined.  This
involves finding bright blobs in the image, and computing each blob's center of
mass to get a point ($x, y$).
Through some 2D to 3D transformation process involving the camera's lens structure, a 3D point is then
obtained~\cite{tappe:starTrackerDevelopment}.
Let the matrix $A$ represent the transformation from $\kFrame$ to $\iFrame$.
In 1965 Gracie Wahba first formulated the attitude determination
problem~\cite{wahba:attitudeEstimationProblem} as
finding the optimal $A$ that minimizes the loss function:
\begin{equation}
    L(A) = \frac{1}{2} \sum_j^n \vv{w_j} \left\| \vv{I_j} - A\vv{K_j} \right\|^2
\end{equation}
where $\vv{w_j}$ represents a nonnegative weight associated with the noise
between the query points $\vv{I_j}$ in the body frame and the matching points $\vv{K_j}$
from the database in the inertial frame. The \textit{TRIAD method} (short for
TRIaxial Attitude Determination) is typically used as a closed form
solution~\cite{markley:attitudeDeterminationTwoVectors} to Wahba's problem after
the evaluating the constellation query to retrieve the matching set of points
from the database.
Hence, constellation queries do not deal with finding the
transformation matrix $A$, but focus on finding the sets of points, i.e.,
constellations, in a database of known points (in a standard
coordinate system) that best matches a set of query points (in
the image coordinate system) in the query image.

This paper is an empirical survey of six existing constellation query processing
strategies in the literature and in the industry. The six strategies are: the
Angle strategy, the Interior Angle strategy, the Spherical Triangle strategy,
the Planar Triangle strategy, the Pyramid strategy, and the Composite
Pyramid strategy. Our contributions are as follows:
\begin{itemize}
\item We developed a unified framework for describing the six existing
constellation query processing strategies/algorithms
\item We implemented all six existing constellation query processing algorithms
\item We performed extensive empirical evaluation of the six constellation query processing
algorithms using the XXXX database and synthetically generated query images.  
\end{itemize} 
To the best of our knowledge, no systematic survey exists for constellation
query processing algorithms, no attempts have been made to study them in a
unified framework, and no empirical evaluation of the algorithms have been done
even though many of these algorithms have been deployed in spacecrafts and
satellites. Our analysis focus on a hardware-independent comparison of
constellation query processing strategies and does not address the preprocessing
of an image to identifying blobs.

%For $n \!>\! 2$, Wahba's problem exists as an optimization problem.
%In the $n\seq2$ case though, the \textit{TRIAD method} (short for TRIaxial Attitude Determination) exists as a
%closed form solution~\cite{markley:attitudeDeterminationTwoVectors}.
%This algorithm starts by constructing two sets of basis vectors: one attached to the body referential (two
%observations in the body frame) $\left[ \vv{t_{1I}} \ \vv{t_{2I}} \ \vv{t_{3I}} \right]$ and another attached to
%the inertial referential (two observations in the inertial frame) $\left[ \vv{t_{2I}} \ \vv{t_{2K}} \ \vv{t_{3K}}
%\right]$~\cite{benet:swisscubeAttitudeDetermination,black:passiveAttitudeDetermination}.
%This is known as the triad frame:
%\begin{alignat}{4}
%    \vv{t_{1I}} &= \frac{\vv{v_1}}{\left| \vv{v_1} \right|} &\vv{t_{2I}} &{}={}&
%    \frac{\vv{u_1}}{\left| \vv{u_1} \right|} \ \ \ \ \ \ \  \\
%    \vv{t_{2I}} &= \frac{\vv{v_1} \times \vv{v_2}}{\left| \vv{v_1} \times \vv{v_2} \right|} \ \ \ \ \ \ \ \
%        &\vv{t_{2K}} &{}={}& \frac{\vv{u_1} \times \vv{u_2}}{\left| \vv{u_1} \times \vv{u_2} \right|} \\
%    \vv{t_{3I}} &= \vv{t_{1I}} \times \vv{t_{2I}} &\vv{t_{3K}} &{}={}& \vv{t_{2I}} \times \vv{t_{2K}}
%\end{alignat}
%Getting from frame $\kFrame$ to $\iFrame$ now simplifies to multiplication of the triad frame base change matrices:
%\begin{equation}
%    A =
%    \begin{bmatrix}
%        \vv{t_{1K}} & \vv{t_{2K}} & \vv{t_{3K}}
%    \end{bmatrix}
%    \begin{bmatrix}
%        \vv{t_{1I}} & \vv{t_{2I}} & \vv{t_{3I}}
%    \end{bmatrix}^T
%\end{equation}

%\begin{subequations}
%Relative to our solar system, the majority of bright stars ($m \!<\! 6.0$, or visible from the Earth with the naked
%eye) do not visibly move.
%Relative to our solar system, the majority of stars visible from Earth with the naked eye do not visibly move.
%    To align these stars with the ones in the catalog, the inverse Mercator mapping is used [CITE ME]:
%    \begin{align}
%        x &= k \cos\left( \frac{x}{R}  \right) \cos\left(2 * \arctan\left(\exp\left(\frac{y}{R}\right)\right) -
%            \frac{\pi}{2}\right) \\
%        y &= k \cos\left( \frac{x}{R}  \right) \sin\left(2 * \arctan\left(\exp\left(\frac{y}{R}\right)\right) -
%            \frac{\pi}{2}\right) \\
%        z &= k \sin\left( \frac{x}{R}  \right)
%    \end{align}

The rest of the paper is organized as follows. Related work is presented in the
next section. Section~\ref{sec:starIdentificationMethods} describes the different
constellation query processing strategies using our unified framework.
Section~\ref{sec:empiricalEvaluation} presents our experimental evaluation of
the different strategies. Section~\ref{sec:conclusion} draws conclusions.
