\section{Conclusion}\label{sec:conclusion}
Attitude determination is a vital part of all spacecraft missions.
Previously, devices such as magnetometers and Sun sensors would give one observation each in both the body and inertial
frames.
The advantage of star trackers is the presentation of multiple observations with just one device.
Star identification is the process associated that maps observations found in the body frame (i.e.\ the image) to the
inertial frame (i.e.\ the catalog).

Gottlieb's Angle method, Liebe's Dot Angle method, Motari's Pyramid method, Cole and Crassidis's Spherical and
Planar Triangle method, and Toloei's Composite Pyramid method were adjusted to fit a general identification flow
and were analyzed in terms of their query, reduction, and identification steps.
Portions that were interchangeable amongst all methods such as database access and centroid determination were
normalized or removed to focus on the star identification aspect itself.

In all experiments, the Pyramid method ranked first (or close to first) in running time and fairly high in accuracy.
The average short running time is a result of the fast query time which despite its larger time complexity, runs as
fast as the Angle method with a much smaller time complexity.
The Pyramid method is the least sensitive to hyperparameter changes in terms of query size response and accuracy
response.
This method produced the lowest number of catalog candidate sets for a given query, and is the most tolerant of
both Gaussian noise and false stars.
The only reason one would not use this method is if an average image does not present at least four stars.
Each image had a field-of-view of 20 degrees, yet there exists a numerous amount of star trackers with a field-of-view
of 5 degrees.

The Angle method always followed or barely beat the Pyramid method in terms of running time, but ranked fairly
low in terms of overall accuracy.
Again, this running time is a result of the fast query time.
Both the Pyramid and Angle method use the same small catalog, which is roughly 35 times as smaller than the next
largest catalogs (Spherical and Planar Triangle).
A smaller catalog means that the search will be faster.
This method can be thought of as the most basic approach to the star identification problem, which explains its speed
and its lack of accuracy.

The Dot Angle method is the most sensitive to hyperparameter changes, but had a larger ideal region than the other
methods.
Despite having the largest catalog (~105 times larger than the Angle/Pyramid catalog), the time to query is actually
faster than that of the Spherical and Planar Triangle methods.
This method also has a smaller time complexity than the triangle methods, placing its running time between the fast
Pyramid/Angle methods and slow triangle methods.
The accuracy of this method is higher than the Angle method, but is not as tolerant to various noise as the Pyramid
method is.

The Spherical and Planar Triangle methods were always 4th and 5th in running time, and roughly tied for accuracy
with the Dot Angle method.
Catalog queries involving triangular features were measured to be 100 times slower than the the Pyramid/Angle methods.
This is a heavy performance hit.
Despite the pivoting reduction process to reduce the number of queries, these methods were always slower than
their angular feature counterparts.

This pivoting process is essential to reducing the running time of the triangle methods, which is why the Composite
Pyramid is the least accurate and slowest of all the methods tested here.
Given a slow query time, an overly aggressive $\{ R \mid 1 = |R| \}$ reduction filter instead of the pivoting process,
and an additional filter in \Call{CompositeVerify}{}, this method is not efficient with its queries.

Overall, the Pyramid method is the most accurate under various noise and produces a result the fastest.
If one is working with a small field-of-view with limited stars to choose from, then the Dot Angle method would be
the next logical choice for retaining accuracy.
If speed is a factor as well, then the Angle method should suffice.