\section{Star Identification Methods}\label{sec:starIdentificationMethods}

\subsection{Generic Identification Method}\label{subsec:genericIdentificationMethod}
Each identification method is presented with an image $I$ of all the stars in the image reference frame. The goal of
each method is to find some mapping between the \textit{subset} of the image star vectors $b$ and a subset of the
catalog star vectors $r$. This mapping will be denoted as $a$, for alignment. Complete identification of the all stars
in each image is not the focus.

Every algorithm starts with some combination $c$ from all possible $k$ combinations of $|I|$ stars $C_k^{|I|}$, where
$k = $ the size of the image subset that specific identification method uses. A set of stars from the image $b$ is
selected using the current combination $c$. For an identification method that uses $k=2$ stars to determine the
alignment in an image of $|I|=4$ stars, the combinations of $I$ are:
\begin{equation}
    C_2^4 \text{ combinations of } I = \{(I_0, I_1), (I_0, I_2), \ldots, (I_3, I_4)\}
\end{equation}

Using certain features of $b$, a set of catalog star sets $R$ is obtained. $R$ is a list of catalog star candidates as
to which $b$ could map to. Referencing the same $k=2$ identification method as before, an image subset $b = (I_0, I_1)$
may yield the candidates in~\autoref{eq:catalogCandidateExample}:
\begin{equation}
    \label{eq:catalogCandidateExample}
    R = \{ (r_{104}, r_{899}), (r_{7622}, r_{7771}), \ldots) \}
\end{equation}

Through some filter process, a single set $r$ from $R$ is then selected. If all of these steps are successful, then the
alignment $a$ between $r$ and $b$ is determined and returned. Otherwise, a new combination $c$ is selected and the
entire process is repeated. This process is detailed
in~\autoref{figure:genericIdentificationMethodFlowchart}.

\begin{figure}
    \centering{
    % Style for process block.
\tikzstyle{process} = [rectangle, text width=3cm, minimum width=3cm, minimum height=1cm,text centered, draw=black,
fill=orange!30]

% Style for terminal block.
\tikzstyle{terminal} = [rectangle, text width=1.7cm, minimum width=1.7cm, minimum height=1.7cm,text centered,
draw=black, fill=red!30]

% Style for decision block.
\tikzstyle{decision} = [diamond, text width=2cm, minimum width=2.5cm, minimum height=2.5cm,text centered, draw=black,
fill=green!30, inner sep=-10pt]

% Style for line.
\tikzstyle{line} = [draw, -latex']

\begin{tikzpicture}[node distance=1.8cm]
    \node[scale=1](getImage)[terminal]{Get Camera Image};
    \node[scale=1](pickQueryStars) [process, left of=getImage, xshift = -1.7cm] {Pick $k$ Image Stars};
    \node[scale=1](searchCatalog)[process, below of=pickQueryStars] {Search Catalog};
    \node[scale=1](confidentInCatalog)[decision, below of=searchCatalog, yshift=-0.5cm] {$|R| > 0$?};
    \node[scale=1](filterCandidates)[process, below of=confidentInCatalog, yshift=-0.5cm] {Filter Candidates};
    \node[scale=1](confidentAfterFilter)[decision, below of=filterCandidates, yshift=-0.5cm] {Confident in $r$?};
    \node[scale=1](findMap)[process, below of=confidentAfterFilter, yshift=-0.5cm]{Identify};
    \node[scale=1](confidentInMap)[decision, below of=findMap, yshift=-0.5cm] {Confident in $a$?};
    \node[scale=1](returnMap)[terminal, right of=confidentInMap, xshift = 1.7cm] {Return Map};

    \draw[->,>=stealth](getImage) -- node[scale=1.3, yshift=-0.3cm]{$I$}(pickQueryStars);
    \draw[->,>=stealth] (pickQueryStars) -- node[scale=1.3, xshift=0.5cm]{$b$}(searchCatalog);
    \draw[->, >=stealth] (searchCatalog) -- node[scale=1.3, xshift=0.5cm, yshift=-0.15cm]{$R$}(confidentInCatalog);
    \draw[->, >=stealth] (confidentInCatalog) -- node[anchor=east, yshift=0.1cm]{Yes}(filterCandidates);
    \draw[->, >=stealth] (filterCandidates) -- node[scale=1.3, xshift=0.5cm, yshift=-0.15cm]{$r$}(confidentAfterFilter);
    \draw[->, >=stealth] (confidentInMap) -- node[xshift=-0.05cm, yshift=0.2cm]{Yes}(returnMap);
    \draw[->, >=stealth] (confidentAfterFilter) -- node[anchor=east, yshift=0.1cm]{Yes} (findMap);
    \draw[->, >=stealth] (findMap) -- node[scale=1.3, xshift=0.5cm, yshift=-0.15cm]{$a$}(confidentInMap);

    \draw[->, >=stealth] (confidentInCatalog.west) -- ++(-1.1cm, 0cm) node[anchor=south, xshift=0.5cm]{No}
    |- (pickQueryStars.west);
    \draw[->, >=stealth] (confidentAfterFilter.west) -- ++(-1.1cm, 0cm) node[anchor=south, xshift=0.5cm]{No}
    |- (pickQueryStars.west);
    \draw[->, >=stealth](confidentInMap.west) -- ++(-1.1cm, 0cm) node[anchor=south, xshift=0.5cm]{No}
    |- (pickQueryStars.west);
\end{tikzpicture}
    \caption{
    Flowchart depicting a general star identification process. Given an image $I$, this processes returns an alignment
    $a$ mapping some subset of the input $b$ to a subset of the catalog $r$.
    } \label{figure:genericIdentificationMethodFlowchart}
    }
\end{figure}

% Leaving this out for now. The flowchart should explain the process better.
%\begin{algorithm}[H]
%    \caption{Generic Star Identification Method} \label{algorithm:genericStarIdentification}
%    \begin{algorithmic}[1]
%        \Procedure{Identify}{}
%        \State $I \gets $ all stars from image
%        \For{$c \in r_k^{|I|}$}
%        \State $b \gets \{I(c_1), I(c_2), I(c_3), \dots, I(r_k)\}$
%        \State $R \gets $ catalog star sets, each set of size $=k$
%        \State $r \gets $ a single set from $R$ that matches $b$
%        \State $a \gets $ an alignment between $b$ and $r$
%        \\
%        \If{the steps above are successful}
%        \State \textbf{return} $a$
%        \EndIf
%        \EndFor
%        \EndProcedure
%    \end{algorithmic}
%\end{algorithm}

% Citation: https://digitalcommons.usu.edu/cgi/viewcontent.cgi?article=2723&context=etd AND Gottlieb.
\subsection{Gottlieb's Angle Method}\label{subsec:gottlieb'sAngleMethod}
In 1978, Gottlieb (citation here) developed the Polygon Angular Matching method. Starting with an image $I$, two stars
$b = (b_1, b_2)$ are selected arbitrarily. The corresponding angular separation between each of these stars from a
defined observer is computed, which is denoted as $\theta^{12}$. All possible pairs $(r_i, r_j)$ in the catalog are then
selected such that condition~\eqref{eq:angleRequirement} holds. The set that holds these pairs is denoted as $R$.

\begin{align}
    \label{eq:angleRequirement}
    \begin{split}
        | \theta(r_i, r_j) - \theta^{12} | < 3 \sigma
    \end{split}
\end{align}
$\sigma$ represents the deviation of the uncertainty between the star sensor measurements and the points defined in the
catalog. Assuming the noise follows a Gaussian distribution, it follows that 99.7\% of all true pairs will be within
this range.

If there exists more than one catalog pair after this reduction, then this process is repeated for
$\theta^{13}, \theta^{14}, \dots,$ and so on until a unique pair is found or all possible pairs in the image
have been exhausted. To determine an alignment, a \textit{direct match test} (DMT) is performed.

\begin{algorithm}
    \caption{Angle Identification Method} \label{algorithm:angleIdentification}
    \begin{algorithmic}[1]
        \Procedure{Identify}{}
        \State $I \gets \text{all stars} \text{ from image}$
        \For{$i \gets 1 \text{\textbf{ to }} |I|$}
        \For{$j \gets i + 1 \text{\textbf{ to }} |I| - 1$}
        \State $b \gets (b_i, b_j)$
        \State $R \gets $ catalog pairs that meet~\eqref{eq:angleRequirement} with $b$
        \\
        \If{$|R| = 1$}
        \State $r \gets $ singular pair inside $R$
        \State $a \gets $ \Call{DMT}{$b, r$}
        \State \textbf{return} $a$
        \EndIf
        \EndFor
        \EndFor
        \EndProcedure
    \end{algorithmic}
\end{algorithm}

Tappe (citation here) specifies this DMT method, which extracts an attitude after running this candidate finding step.
Given an image star pair $(b_i, b_j)$ and a catalog star pair $(r_i, r_j)$, an alignment is proposed:
\begin{equation}
    a_1 = (b_i \rightarrow r_i, b_j \rightarrow r_j)
\end{equation}

Wahba's problem (extracting an attitude given vector observations in two coordinate systems) is then solved using the
TRIAD method. This gives a rotation $q_1$ between the image and catalog frames. This process is repeated for the other
possible alignment to obtain a second rotation $q_2$:
\begin{equation}
    a_2 = (b_i \rightarrow r_j, b_j \rightarrow r_i)
\end{equation}

The most likely attitude is determined by predicting which entries in the catalog represent stars in the image using
$q_1$ and $q_2$. The alignment with the most correctly predicted stars is then returned as the resulting attitude.
The general form for this function (accepting an $B$ and $R$ of arbitrary size) is described
in algorithm~\autoref{algorithm:angleHelper}.

\begin{algorithm}
    \caption{Functions for Angle Method} \label{algorithm:angleHelper}
    \begin{algorithmic}[1]
        \Function{DMT}{$b, r$}
        \State $A \gets $ all possible permutations between $b$ and $r$
        \State $M \gets \emptyset$
        \For {$a \in A$}
        \State $q \gets $ \Call{TRIAD}{$a$}
        \State $M \gets M ||$ (stars correctly predicted with $q$)
        \EndFor
        \\
        \State \textbf{return} alignment $a$ associated with \Call{Max}{$M$}
        \EndFunction
    \end{algorithmic}
\end{algorithm}

\subsection{Liebe's Interior Angle Method}\label{subsec:liebe'sInteriorAngleMethod}
In 1995, Liebe (citation here) developed the Liebe Star ID method. Starting with an image $I$, a central star $b_c$ is
selected arbitrarily. The two closest stars in the image to the central star are selected next, denoted as $b_1$ and
$b_2$. Three features are then computed: the angular separation between $b_1$ and $b_c$, the angular separation between
$b_2$ and $b_c$, and the interior angle between $b_1$ and $b_2$ with $b_c$ at the vertex. These are denoted as
$\theta^1, \theta^2,$ and $\phi$ respectively.

The additional constraint that $\theta^1 < \theta^2$ is imposed before proceeding. If this is not true, then stars
$b_1$ and $b_2$ are swapped and this process is repeated. By adding this restriction to the catalog search, a star
alignment procedure (e.g.\ Tappe's direct match test) is no longer required.

All possible \textit{trios} $R$ in the catalog are then selected such that all of
condition~\eqref{eq:interiorAngleRequirement} hold:
\begin{align}
    \label{eq:interiorAngleRequirement}
    \begin{split}
        |\theta(r_i, r_c) - \theta^1| < 3 \sigma_{\theta}
        \\
        |\theta(r_j, r_c) - \theta^2| < 3 \sigma_{\theta}
        \\
        |\phi(r_i, r_j, r_c) - \phi| < 3 \sigma_{\phi}
        \\
        \theta(r_i, r_c) < \theta(r_j, r_c)
    \end{split}
\end{align}

$\sigma_{\theta}$ and $\sigma_{\phi}$ represent the deviation of measurement-catalog uncertainty of angular separations
and interior angular separations respectively. \textit{Note that the $R$ in this procedure is distinct from the $R$ in
the previous Angle method.}

The star trios in $R$ represent potential catalog maps for the image star trio $(b_1, b_2, b_c)$. Liebe's original
method states that this process should be repeated for all stars in the image, meaning that all stars will be the $b_c$
at one point. By the end, each star in the image will have accrued a set of possible catalog matches $P$. The complete
$B \rightarrow R$ map is found by picking the most frequent catalog star appearing in $P$.

To more closely follow the generic star identification flow, $P$ will not be stored and a only one $b_c$ choice will be
needed to acquire a total match. If a confident match is not found by the first $b_c$ star, then the search process
will be repeated until such a match is found. This entire method is described in
algorithm~\autoref{algorithm:interiorAngleIdentification}.

\begin{algorithm}
    \caption{Interior Angle Identification Method} \label{algorithm:interiorAngleIdentification}
    \begin{algorithmic}[1]
        \Procedure{Identify}{}
        \State $I \gets \text{all stars } \text{ from image}$
        \For{$i \gets 1 \text{\textbf{ to }} |I|$}
        \For{$j \gets i + 1 \text{\textbf{ to }} |I| - 1$}
        \For{$k \gets j + 1 \text{\textbf{ to }} |I| - 2$}
        \State $b \gets (b_i, b_j)$
        \State $R \gets $ catalog trios that meet~\eqref{eq:interiorAngleRequirement} with $b$
        \\
        \If{$|R| = 1$}
        \State $(r_i, r_j) \gets $ singular pair inside $R$
        \State $a \gets (b_i \rightarrow r_i, b_j \rightarrow r_j, b_k \rightarrow r_k)$
        \State \textbf{return} $a$
        \EndIf
        \EndFor
        \EndFor
        \EndFor
        \EndProcedure
    \end{algorithmic}
\end{algorithm}

\subsection{Cole and Crassidus's Spherical Triangle Method}\label{subsec:coleAndCrassidus'sSphericalTriangleMethod}
In 2004, Cole and Crassidus (citation here) developed the Spherical Triangle method. Starting with an image $I$, three
stars $b = b_1, b_2, b_3)$ are selected arbitrarily. Treating the trio as a spherical triangle, the spherical area and
moment are computed. This is denoted as $a^{123}$ and $\imath^{123}$ respectively. Similar to Gottlieb's method, star
\textit{trios} $R$ are selected from the catalog such that the all of condition~\eqref{eq:triangleRequirement} hold:
\begin{align}
    \begin{split}
        \label{eq:triangleRequirement}
        | a(r_i, r_j, r_k) - a^{123} | < 3 \sigma_a
        \\
        | \imath(r_i, r_j, r_k) - \imath^{123} | < 3\sigma_{\imath}
    \end{split}
\end{align}
$\sigma_a$ and $\sigma_{\imath}$ represent the deviation of measurement-catalog uncertainty of spherical areas and
moments respectively.

If there exists more than one catalog trio, then a \textit{pivot} is performed. The pivoting process is repeated until
a unique match in $R_{t=1}$ is found, or all possible iterations of the third image star are exhausted. The entire
catalog search procedure is repeated until a unique catalog trio is found, or all trios in the image have been used.
This procedure is described in algorithm~\autoref{algorithm:triangleIdentification}.

\begin{algorithm}
    \caption{Triangle Method Identification} \label{algorithm:triangleIdentification}
    \begin{algorithmic}[1]
        \Procedure{Identify}{}
        \State $I \gets \text{all stars} \text{ from image}$
        \For{$i \gets 1 \text{\textbf{ to }} |I|$}
        \For{$j \gets i + 1 \text{\textbf{ to }} |I| - 1$}
        \For{$k \gets j + 1\text{\textbf{ to }} |I| - 2$}
        \State $R \gets$ \Call{Pivot}{$b_i, b_j, b_k, \emptyset$}
        \If{$R \neq \emptyset$}
        \State \textbf{return} $R$
        \EndIf
        \EndFor
        \EndFor
        \EndFor
        \EndProcedure
    \end{algorithmic}
\end{algorithm}

The pivot procedure starts by setting $R$ to $R_{t=1}$, beginning the pivot with the trio set that was just queried
for. A second set of trios $R_{t=2}$ is retrieved using $\bar{b} = (b_1, b_2, b_4)$, keeping $b_1$ and $b_2$ constant
but changing the third image star. All star trios in $R_{t=1}$ that do not match a trio in $R_{t=2}$ by
\textit{two stars} (a partial match) are removed from $R_{t=1}$. Note that the \Call{Pivot}{$b_i, b_j, b_k, R_1$} call
inside algorithm~\autoref{algorithm:triangleIdentification} defines $R_1 = \emptyset$, which requires a "check" on
lines 2 and 3 of algorithm~\autoref{algorithm:triangleHelper}.

\begin{algorithm}
    \caption{Functions for Triangle Identification} \label{algorithm:triangleHelper}
    \begin{algorithmic}[1]
        \Function{PartialMatch}{$R_1, R_t$}
        \If {$R_t = \emptyset$} \Comment $t = 1$, no previous set.
        \State \textbf{return} $R_1$
        \EndIf
        \\
        \State $\bar{R_1} \gets \emptyset$
        \ForAll {$v \in R_t$}
        \ForAll {$u \in R_1$}
        \If{two stars in $v$ exist in $u$}
        \State $\bar{R_1} \gets \bar{R_1} || v$
        \State \textbf{break}
        \EndIf
        \EndFor
        \EndFor
        \State \textbf{return} $\bar{R_1}$
        \EndFunction
        \\
        \Function{Pivot}{$b_i, b_j, b_k, R_1$}
        \State $b \gets (b_j, b_j, b_k)$
        \State $R_t \gets $ catalog trios that meet~\eqref{eq:triangleRequirement} with $b$
        \State $R_1 \gets $ \Call{PartialMatch}{$R_t, R_1$}
        \\
        \If{$|R_1| = 1$}
        \State \textbf{return} $R_1$
        \ElsIf{$|R_1| = 0$}
        \State \textbf{return} $\emptyset$
        \Else
        \State $\hat{b_k} \gets \text{an unused star in this pivot}$
        \State \textbf{return} \Call{Pivot}{$b_i, b_j, \hat{b_k}, R_1$}
        \EndIf
        \EndFunction
    \end{algorithmic}
\end{algorithm}

Cole and Crassidus do not specify alignment determination steps, so Tappe's DMT process is used to complete the star
identification process. Given an image star trio $(b_i, b_j, b_k)$ and a catalog star trio $(r_i, r_j, r_k)$, an
alignment is proposed:
\begin{equation}
    a_1 = (s_i \rightarrow r_i, b_j \rightarrow r_j, b_k \rightarrow r_k)
\end{equation}

The TRIAD method only uses two vector observations from each frame, meaning that the $b_k \rightarrow r_k$ map is
disregarded as the first rotation $q_1$ is computed. This process is repeated for all 5 other possible alignments to
get $q_2, q_3, \dots, q_6$.
\begin{description}
    \item [$a_1 = $] $(s_i \rightarrow r_i, b_j \rightarrow r_j, b_k \rightarrow r_k)$
    \item [$a_2 = $] $(s_i \rightarrow r_i, b_j \rightarrow r_k, b_k \rightarrow r_j)$
    \item [$a_3 = $] $(s_i \rightarrow r_j, b_j \rightarrow r_i, b_k \rightarrow r_k)$
    \item [$a_4 = $] $(s_i \rightarrow r_j, b_j \rightarrow r_k, b_k \rightarrow r_i)$
    \item [$a_5 = $] $(s_i \rightarrow r_k, b_j \rightarrow r_i, b_k \rightarrow r_j)$
    \item [$a_6 = $] $(s_i \rightarrow r_k, b_j \rightarrow r_j, b_k \rightarrow r_i)$
\end{description}

For all six attitudes, the $a$ yielding the most correctly predicted stars is returned as the resulting attitude.

\subsection{Cole and Crassidus's Planar Triangle Method}\label{subsec:coleAndCrassidus'sPlanarTriangleMethod}
In 2006, Cole and Crassidus (citation here) developed the Planar Triangle method. This is identical to their Spherical
Triangle method, with the exception that each image trio $b = (b_i, b_j, b_k)$ is represented as a planar triangle
instead of a spherical one.

Computing the spherical moment requires the use of recursion, which could be costly in slower hardware to obtain more
precision. The Planar Triangle method avoids this by computing the planar area and moment instead, which do not require
this recursive step.

% Mortari introduced the use of \textit{search-less} catalog access using the $k$-vector approach, but this will not be
% discussed in this paper.
\subsection{Mortari's Pyramid Star Identification Method}\label{subsec:mortari'sPyramidStarIdentificationMethod}
In 2004, Mortari (citation here) developed the Pyramid method. Starting with an image $I$, three stars
$b = (b_1, b_2, b_3)$ are selected to avoid the persistence of false stars. This is specified in lines 19 to 23 in
algorithm~\autoref{algorithm:pyramidIdentification}. The corresponding angular separation between each distinct
permutation of the three is computed, denoted as $\theta^{12}, \theta^{13}, \theta^{23}$. All possible
\textit{pair sets} $R^{12}, R^{13}, R^{23}$ are selected from the catalog such that
condition~\eqref{eq:angleRequirement} holds for each respective $\theta$. The common stars between each of these
pairs inside $R$ is then determined, yielding $T^1, T^2, $ and $T^3$. If there exists a sole element inside each
$T$, then a secondary verification involving guessing a nearby star is performed. Once this verification step is
past, an alignment is then returned. This entire process is described in
algorithm~\autoref{algorithm:pyramidIdentification}.

\begin{algorithm}
    \caption{Pyramid Identification Method} \label{algorithm:pyramidIdentification}
    \begin{algorithmic}[1]
        \Function {FindCatalogStars} {$b, I$}
        \State $R^{ij} \gets$ catalog pairs that meet~\eqref{eq:angleRequirement} with ($b_i, b_j$)
        \State $R^{jk} \gets$ catalog pairs that meet~\eqref{eq:angleRequirement} with ($b_i, b_j$)
        \State $R^{ik} \gets$ catalog pairs that meet~\eqref{eq:angleRequirement} with ($b_i, b_k$)
        \\
        \State $T^i \gets $ \Call{Common}{$R^{ij}, R^{ik}$}
        \State $T^j \gets $ \Call{Common}{$R^{ij}, R^{jk}$}
        \State $T^k \gets $ \Call{Common}{$R^{ik}, R^{jk}$}
        \\
        \If{$|T^i| = 1 $ \textbf{and} $|T^j| = 1 $ \textbf{and} $|T^k| = 1$}
        \State $r \gets $ singular stars in $T_i, T_j, T_k$
        \If{\Call{PyramidVerify}{$r, b, I$} is \textbf{true}}
        \State \textbf{return} $r$
        \EndIf
        \EndIf
        \State \textbf{return} $\emptyset$
        \EndFunction
        \\
        \Procedure{Identify}{}
        \State $I \gets \text{all stars } s \text{ from image}$
        \For{$dj \gets 1 \text{\textbf{ to }} |I| - 2$}
        \For{$dk \gets 1 \text{\textbf{ to }} |I| - 1 - dj$}
        \For{$k \gets i \text{\textbf{ to }} |I| - dj - dk$}
        \State $j \gets i + dj$
        \State $k \gets j + dk$
        \\
        \State $b \gets (b_i, b_j, b_k)$
        \State $r \gets$ \Call{FindCatalogStars}{$b, I$}
        \If{$r \neq \emptyset$}
        \State $a \gets (b_i \rightarrow r_i, b_j \rightarrow r_j, b_k \rightarrow r_k)$
        \State \textbf{return} $a$
        \EndIf
        \EndFor
        \EndFor
        \EndFor
        \EndProcedure
    \end{algorithmic}
\end{algorithm}

The common stars $T^a$ in two sets of star \textit{pairs} $R^{ab}$ and $R^{ac}$, is found by "flattening" the list of
pairs and taking the union of the two. If $R^{ab} = \{ (1, 2), (2, 3) \}$ and $R^{ac} = \{ (2, 5), (7, 3) \}$, then the
common stars for $a$ is $T^a = \{2, 3\}$.
\begin{align}
    \label{eq:commonStarsPyramid}
    \begin{split}
        T^a = \{ r | r \in R \in R^{ab} \} \cap \{ r | r \in R \in R^{ac} \}
    \end{split}
\end{align}

%The function \Call{Common}{$R^{ab}, R^{ac}, E$} in algorithm~\autoref{algorithm:pyramidHelpers} also specifies an $E$
%parameter, which acts as a "exclusion" set. Once a common star is found (e.g.\ $T^i$), it follows that the next
%common star to be found (e.g.\ $T^j$) should not include this previously found star.

The second verification step is denoted as \Call{PyramidVerify}{$r, b, I$} in
algorithm~\autoref{algorithm:pyramidHelpers}. This function starts by selecting a nearby star to $b$ in the image
$b_e$. The angular separation between each star in $b$ is computed, generating $\theta^{ie}, \theta^{je},
\theta^{ke}$. Again, three sets of catalog pairs $R^{ie}, R^{je}, R^{ke}$ are selected such that
condition~\eqref{eq:angleRequirement} holds for each respective $\theta$. If the common star between all $R$ sets
exists near the given $r$ set, then the test past. Otherwise, the test fails.

\begin{algorithm}
    \caption{Functions for Pyramid Identification} \label{algorithm:pyramidHelpers}
    \begin{algorithmic}[1]
        \Function{Common}{$R^{ab}, R^{ac}$}
        \State \textbf{return} all stars that meet~\eqref{eq:commonStarsPyramid} with $R^{ab}, R^{ac}$
        \EndFunction
        \\
        \Function{PyramidVerify}{$r, b, I$}
        \State $b_e \gets $ star in image $I$ not in $b$
        \State $R^{ie} \gets$ catalog pairs that meet~\eqref{eq:angleRequirement} with $(b_i, b_e)$
        \State $R^{je} \gets$ catalog pairs that meet~\eqref{eq:angleRequirement} with $(b_j, b_e)$
        \State $R^{ke} \gets$ catalog pairs that meet~\eqref{eq:angleRequirement} with $(b_k, b_e)$
        \\
        \State $T^e \gets $ \Call{Common}{$R^{je}, R^{ke} \cup R^{ie}$}
        \If{$|T^e| = 1$}
        \State $r_e \gets $ singular star in $T^e$
        \If{$r_e$ is near $r$ in catalog}
        \State \textbf{return true}
        \EndIf
        \EndIf
        \State \textbf{return false}
        \EndFunction
    \end{algorithmic}
\end{algorithm}

\subsection{Toloei's Composite Pyramid Method}\label{subsec:toloei'sCompositePyramidMethod}
In 2014, Toloei (citation here) developed the Novel Stars ID method, which retains all of the key aspects of Motari's
Pyramid method but uses the features from Cole and Crassidus's Planar Triangle method for the query and reference steps
instead of angles. Starting with an image $I$, three stars $b = (b_1, b_2, b_3)$ are selected in the same "smart"
manner as the Pyramid method. Treating the trio as a planar triangle, the planar area and moment are computed.
This is denoted as $a^{123}$ and $\imath^{123}$ respectively. Similar to the Planar Triangle method, star
\textit{trios} $R$ are selected from the catalog such that condition~\eqref{eq:triangleRequirement} hold.

Unlike the triangle methods, no pivot is performed. This searching process is repeated until a unique $R$ is found.
Toloei does not specify an attitude determination method, and an alignment cannot be determined using common
stars like the Pyramid method. To bridge this gap, Tappe's DMT process is used in the same manner as the Triangle
methods. With an alignment determined, the secondary verification step involving guessing a nearby star is performed.
If this verification step passes, then an alignment returned. This entire process is described in
algorithm~\autoref{algorithm:compositePyramidIdentification}.

\begin{algorithm}
    \caption{Composite Pyramid Identification Method} \label{algorithm:compositePyramidIdentification}
    \begin{algorithmic}[1]
        \Function {FindCatalogStars} {$b, I$}
        \State $R \gets $ catalog trios that meet~\eqref{eq:triangleRequirement} with $b$
        \\
        \If{$|R| = 1$}
        \State $r \gets $ singular trio inside $R$
        \State $a \gets $ \Call{DMT}{$b, r$}
        \If{\Call{CompositeVerify}{$a, I$} is \textbf{true}}
        \State \textbf{return} $a$
        \EndIf
        \EndIf
        \State \textbf{return } $\emptyset$
        \EndFunction
        \\
        \Procedure{Identify}{}
        \State $I \gets \text{all stars } s \text{ from image}$
        \For{$dj \gets 1 \text{\textbf{ to }} |I| - 2$}
        \For{$dk \gets 1 \text{\textbf{ to }} |I| - 1 - dj$}
        \For{$k \gets i \text{\textbf{ to }} |I| - dj - dk$}
        \State $j \gets i + dj$
        \State $k \gets j + dk$
        \\
        \State $b \gets (b_i, b_j, b_k)$
        \State $a \gets$ \Call{FindCatalogStars}{$b, I$}
        \If{$a \neq \emptyset$}
        \State \textbf{return} $a$
        \EndIf
        \EndFor
        \EndFor
        \EndFor
        \EndProcedure
    \end{algorithmic}
\end{algorithm}

To second verification step is denoted as \Call{CompositeVerify}{$b, r, a, I$} in
algorithm~\autoref{algorithm:compositePyramidHelpers}. Like the Pyramid verification function, this verification starts
by selecting a nearby star to $b$ in the image $b_e$. The planar area and moment are computed for

The angular separation
between each star in b is computed, generating θ ie ,θ je ,θ ke .
Again, three sets of catalog pairs R ie ,R je ,R ke are selected
such that condition (3) holds for each respective θ. If the
common star between all R sets exists near the given r set,
then the test past. Otherwise, the test fails.

The same function \Call{Common}{$R^{ab}, R^{ac}$} from the
Pyramid method is used to determine what the
% TODO: need to account for the fact that there exists two common stars here.... Need to add exlusion set !!!!
\begin{algorithm}
    \caption{Functions for Composite Pyramid Identification} \label{algorithm:compositePyramidHelpers}
    \begin{algorithmic}[1]
        \Function{CompositeVerify}{$b, r, a, I$}
        \State $b_e \gets $ star in image $I$ not in $b$
        \State $b^{\star} \gets $ stars in $b$ aligned using $a$
        \State $r^{\star} \gets $ stars in $r$ aligned using $a$

        \State $R^{ije} \gets $ catalog trios that meet~\eqref{eq:triangleRequirement} with $(b^{\star}_i, b^{\star}_j,
        b_e)$
        \State $R^{ike} \gets $ catalog trios that meet~\eqref{eq:triangleRequirement} with $(b^{\star}_i, b^{\star}_k,
        b_e)$
        \State $R^{jke} \gets $ catalog trios that meet~\eqref{eq:triangleRequirement} with $(b^{\star}_j, b^{\star}_k,
        b_e)$
        \State $T^e \gets $ \Call{Common}{$R^{ije}, R^{ike} \cup R^{jke}$}
        \If{$|T^e| = 1$}
        \State $r_e \gets $ singular star in $T^e$
        \If{$r_e$ is near $r$ in catalog}
        \State \textbf{return true}
        \EndIf
        \EndIf
        \State \textbf{return false}
        \EndFunction
    \end{algorithmic}
\end{algorithm}

\begin{table*}[ht]
    \centering {
    \caption{
    Overview table of the different identification methods. Each method's image features, reduction process,
    and alignment process is displayed.
    } \label{tab:identificationMethodOverview}
    \begin{tabular}{  m{0.22\linewidth} || m{0.21\linewidth} | m{0.21\linewidth} | m{0.24\linewidth} }
        & \textbf{Image Features} & \textbf{Reduction} & \textbf{Alignment} \\
        \hline \hline
        \textbf{Angle} & $\theta^{ij}$ & Require $|R|=1$ & \Call{DMT}{$b, r$} \\ \hline
        \textbf{Interior Angle} & $\theta^{ij}, \theta^{jc}, \phi$ & Require $|R| = 1$ & Restrict $\theta^{ic},
        \theta^{jc}$ at Query Time \\ \hline
        \textbf{Spherical Triangle} & Spherical $a^{ijk}, \imath^{ijk}$ & \Call{Pivot}{$b_i, b_j, b_k, R_1$} &
        \Call{DMT}{$b, r$} \\ \hline
        \textbf{Planar Triangle} & Planar $a^{ijk}, \imath^{ijk}$ & \Call{Pivot}{$b_i, b_j, b_k, R_1$} &
        \Call{DMT}{$b, r$} \\ \hline
        \textbf{Pyramid} & $\theta^{ij}, \theta^{ik}, \theta^{jk}$ & Require $|R| = 1$ &
        \Call{Common}{$R^{ab}, R^{ac}, F$}, \newline \Call{PyramidVerify}{$r, b, I$} \\ \hline
        \textbf{Composite Pyramid} & Planar $a^{ijk}, \imath^{ijk}$ & Require $|R| = 1$ & \Call{DMT}{$b, r$}, \newline
        \Call{CompositeVerify}{$r, b, a, I$}
    \end{tabular}
    }
\end{table*}