\section{Motivation} 

In 1976, two researchers at JPL published a paper regarding a device that could
detect stars using a CCD camera. The first star tracker was born, and the
authors noted the use of this device for accurate attitude determination. The
approach to attitude determination at the time had been to use two sensors (most
likely cameras to detect the Sun) to draw two vectors to a known reference and
extract an orientation from this. Star trackers allow the use of one sensor to
draw a numerous amount of vectors to a known reference, increasing the accuracy
of the estimated orientation.

Fast forward to today, the number of solutions to the star identification
problem has grown quite a bit. There have been improvements in various aspects
of star tracking, but a good portion of this research has been separate from
each other.

\textit{The goal of this project is to create an optimal complete 
solution for the lost-in-space version of the problem based on strengths of
existing algorithms and empirical observation.} 