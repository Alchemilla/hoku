\tdplotsetmaincoords{60}{110}

% Origin for the secondary coordinate system (body), in spherical coordinates of the inertial.
\pgfmathsetmacro{\rvec}{.8} \pgfmathsetmacro{\thetavec}{30} \pgfmathsetmacro{\phivec}{60}

\begin{tikzpicture}[scale=15,tdplot_main_coords]

    % Draw the inertial frame.
    \coordinate (O) at (0,0,0);
    \node[color=ngreen, anchor=east, fill=white, xshift=-0.3cm] at (0,0,0) {Standard Frame};
    \draw[thick,->,color=ngreen,line width=0.1cm] (0,0,0) -- (0.7,0,0) node[anchor=north east]{$u_1$};
    \draw[thick,->,color=ngreen,line width=0.1cm] (0,0,0) -- (0,0.7,0) node[anchor=north west]{$u_2$};
    \draw[thick,->,color=ngreen,line width=0.1cm] (0,0,0) -- (0,0,0.7) node[anchor=south]{$u_3$};

    % Plot the origin body frame in the inertial frame, show this connection.
    \tdplotsetcoord{P}{\rvec}{\thetavec}{\phivec}
%    \draw[dashed] (O) -- (P) node[anchor=north east, xshift=-0.5cm, yshift=-0.5cm]{$A^\nicefrac{I}{C}$};

    % Setup the body frame (don't draw till the end).
    \tdplotsetrotatedcoords{\phivec}{\thetavec}{0}
    \tdplotsetrotatedcoordsorigin{(P)}

    % Draw the point itself.
    \filldraw[tdplot_rotated_coords, color=jblue] (.205,.205,.205) circle[radius=0.3pt];
%    \draw[tdplot_rotated_coords] (.255,.255,.255) circle[radius=0.3pt];   
    \filldraw[tdplot_rotated_coords, color=jblue] (.255,.205,.225) circle[radius=0.3pt]; 
    \filldraw[tdplot_rotated_coords, color=jblue] (.155,.155,.155) circle[radius=0.3pt]; 
    \filldraw[tdplot_rotated_coords, color=jblue] (.3,.155,.155) circle[radius=0.3pt]; 
    \filldraw[tdplot_rotated_coords, color=jblue] (.35,.01,.1) circle[radius=0.3pt]; 
    \filldraw[tdplot_rotated_coords, color=jblue] (.01,.35,.155) circle[radius=0.3pt]; 
    
%    \node[tdplot_rotated_coords, anchor=west, xshift=1cm, color=dblue!70] at (.2,.2,.2) {};

    % Draw the point in the body frame.
%    \draw[-stealth,color=dblue!70,tdplot_rotated_coords] (0,0,0) -- (.205,.205,.205);
%    \node[color=dblue!70,tdplot_rotated_coords,fill=white,xshift=0.6cm] at (.1,.1,.1) {};
%    \draw[dashed,color=dblue!70,tdplot_rotated_coords] (0,0,0) -- (.2,.2,0);
%    \draw[dashed,color=dblue!70,tdplot_rotated_coords] (.2,.2,0) -- (.2,.2,.2);

    % Draw the point in the inertial frame.
%    \coordinate (Q) at (0.4,0.767,0.91);
%    \draw[-stealth,color=ngreen] (O) -- (Q);
%    \node[color=ngreen,fill=white,xshift=-0.1cm,yshift=-0.1cm] at (0.2,0.3835,0.455) {};
%    \draw[dashed, color=ngreen] (O) -- (0.4,0.767,0);
%    \draw[dashed, color=ngreen] (Q) -- (0.4,0.767,0);

    % Draw the body frame.
    \draw[thick,tdplot_rotated_coords,->,color=dblue!70,line width=0.1cm] (0,0,0) -- (.5,0,0) node[anchor=east]{$v_1$};
    \draw[thick,tdplot_rotated_coords,->,color=dblue!70,line width=0.1cm] (0,0,0) -- (0,.5,0) node[anchor=west]{$v_2$};
    \draw[thick,tdplot_rotated_coords,->,color=dblue!70,line width=0.1cm] (0,0,0) -- (0,0,.5) node[anchor=west]{$v_3$};
    \node[color=dblue!70,tdplot_rotated_coords,anchor=east,xshift=-0.3cm, yshift=0cm, fill=white] at (0,0,0) {Image Frame};

\end{tikzpicture}
