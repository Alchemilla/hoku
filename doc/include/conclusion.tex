\section{Conclusion}\label{sec:conclusion}
In this paper, we discussed six constellation querying strategies and their strengths and weaknesses.
%A unified identification framework was created to describe all strategies for fair analysis.
%Portions that were interchangeable amongst all strategies such as database access and centroid determination were
%normalized or removed to focus on the star identification aspect itself.
%To control the severity of our error, artificial images were generated.

The Angle strategy is the simplest of the six and has the fastest query step, but its runtime is heavily impacted by
the $\abs{R} = 1$ criterion and \Call{DMT}{} process.
The Interior Angle strategy is the fastest running strategy under no noise, but its accuracy is the most sensitive to
varying Gaussian noise and the slowest query step.
The Spherical Triangle strategy's accuracy is the least sensitive to varying Gaussian noise, but is the most sensitive
to varying amounts of false stars.
The Planar Triangle is on average faster than the Spherical Triangle strategy, but is also very sensitive to varying
amounts of false stars.
The Pyramid strategy is the fastest strategy given varying amounts of Gaussian noise \& false stars and is also the most
accurate given varying amounts of spikes, but is not able to achieve $100\%$ average accuracy due to its query step.
The Composite Pyramid strategy does not suffer from this inaccuracy problem, but does not achieve the same consistent
performance of the Pyramid or the triangle strategies due to the number of filters implemented.

Overall, the Pyramid strategy handles both Gaussian noise and false stars the best in a reasonable amount of time.
