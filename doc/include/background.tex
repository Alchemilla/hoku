\section{Introduction}\label{sec:introduction}
With the advent of commercial space industry, we are seeing
an increasing number of spacecrafts, both manned and unmanned,
launched into space. One important function on a spacecraft
is the ability to determine its orientation in space using
the images captured by a camera on the spacecraft - this is
often known as the \textit{Lost-in-space} problem.
For example, consider the design of low Earth
orbit (LEO) spacecrafts. In order for the spacecraft to
point a payload, direct its thrusters, or orient its solar
panels, an accurate \textit{attitude} (another term for
orientation) must be known. There are a few known landmarks
in space where some attitude can be extracted (e.g., the
Earth, the Sun), but this requires constant direction
towards just these objects. Most attitude determination
systems do not limit themselves to using a single object as
a landmark, rather they use multiple stars within the
field of view of their camera to determine their orientation.
The images capture a rectangular region of space -- from an
earth centric perspective, we can think of an image of the
sky at night. Each image is essentially a collection of
bright spots (celestial objects) and their location in the
image coordinate system. To find the orientation of the
spacecraft, we have to match the pattern of celestial
objects in the image to a database of known celectial
objects. We refer to such database queries as 
\textit{Constellation Queries}: Given a set of query points (in
the image coordinate system), find the sets of points, i.e.,
constellations, in a database of known points (in a standard
coordinate system) that best matches the pattern of the
query points. 

Hypothetically, if we could transform the
query points and database points to a common coordinate
system, then constellation queries would be a simple
database lookup for each query point. Unfortunately,
evaluating constellation queries are complicated by several
issues. First the query points and the points in the
database are in two different coordinate system (also known
as reference frames) and the transformation between the two
coordinate system is not known a priori. In fact, attitude
determination \emph{is} finding the transformation between
the two coordinate system. Second, the query
points are affected by transient celestial objects (e.g.,
meteors) resulting in spurious points in the image,
obstructions resulting in missing points, and camera
charateristics resulting in deviations of the query points
true position.

%Ancient mariners could look up at the night sky, point out which stars they were looking at, and navigate across the
%globe without the use of maps.
%\textit{Star identification algorithms} refer to computational approaches to determining which stars are in the sky.
%Given an image of the sky, star identification is matching the bright spots in an image to stars in an astronomical
%catalog.
%The device that performs these computations is the star tracker, much like the navigators on the ship.
%\textit{Lost-in-space} refers to an additional constraint on the problem: the absence of knowing where we took
%the picture and how we pointed the camera.

%There exist roughly $4{,}500$ stars in the sky visible to the human eye.
%For an image of $n$ stars, the naive approach would be compute $C(4{,}500, n)$ combinations from this collection and
%compare each to some subset of stars found in the image.
%For $n\seq 3$, this requires over $10^{10}$ comparisons.
%As an alternative, we sacrifice storage and precision for speed by searching a separate collection which indexes the
%${\sim}4{,}500$ stars by one or more features.
%When this subset is identified, we determine and return the orientation of the image relative to collection
%of ${\sim}4{,}500$ stars.

This research is motivated by a growing difference in the
number of stellar attitude determination methods and
empirical comparison between each of these methods in a more
systematic manner for star tracker development.
Interchangeable factors are abstracted away (camera
hardware, blob detection, etc\ldots) to focus more on how
each query strategy matches stars in an image to stars in a
database. This paper aims to contribute a hardware
independent comparison process, an algorithmic description
of several query strategies, as well as runtime and catalog
access analysis of these strategies under various types of
noise. The process of identifying blobs in an image,
constructing the image coordinate system, and efficiently
querying static databases are not addressed here.

\subsection{Stellar Based Attitude Determination}\label{subsec:stellarBasedAttitudeDetermination}
%Attitude refers to the translation between how one system describes an object compared to how a different system
%describes the same object.
%
%In the context of spacecraft attitude for star identification, there exist three reference frames: the
%\textit{body frame}, the \textit{sensor frame}, and the \textit{inertial frame}.
%The body frame itself is fixed to the structure of the spacecraft, the sensor frame is fixed to the star tracker,
%and the inertial frame refers to some non-accelerating frame in which stellar objects are recorded.
%All observations from the spacecraft exist in the sensor frame, but can easily be rotated to align with the body frame
%(the sensor itself is fixed to the spacecraft chassis).
%Consequently, the body frame is used interchangeably with the sensor frame.
%To describe the craft itself, an inertial frame is required for finding a practical attitude.
%A star observed in the inertial frame is more predictable than the same star observed in a tumbling spacecraft, aiding
%the usage of the attitude with orientation dependent processes.
%Using all three, the goal of attitude determination becomes finding some method of translation between the inertial
%frame and the body frame.

Let $\kFrame$ describe an inertial reference frame and $\iFrame$ describe a body reference
frame~\cite{wie:spaceVehicleDynamics}.
To find a matrix $A$ that describes the basis vectors of $\kFrame$ in terms of $\iFrame$ but accounts
for the noise of each measurement is known as \textit{Wahba's problem}.
First posed by Gracie Wahba in 1965~\cite{wahba:attitudeEstimationProblem},
Wahba's problem states that finding the optimal $A$ is minimizing the loss function below:
\begin{equation}
    L(A) = \frac{1}{2} \sum_j^n \vv{w_j} \left\| \vv{I_j} - A\vv{K_j} \right\|^2
\end{equation}
where $\vv{w_j}$ represents a non negative weight associated with the noise between the observations $\vv{I_j}$
in the body frame and $\vv{K_j}$ in the inertial frame.
For all instances where Wahba's problem appeared, the \textit{TRIAD method} (short for TRIaxial Attitude Determination)
was used as a closed form solution~\cite{markley:attitudeDeterminationTwoVectors}.

%
%For $n \!>\! 2$, Wahba's problem exists as an optimization problem.
%In the $n\seq2$ case though, the \textit{TRIAD method} (short for TRIaxial Attitude Determination) exists as a
%closed form solution~\cite{markley:attitudeDeterminationTwoVectors}.
%This algorithm starts by constructing two sets of basis vectors: one attached to the body referential (two
%observations in the body frame) $\left[ \vv{t_{1I}} \ \vv{t_{2I}} \ \vv{t_{3I}} \right]$ and another attached to
%the inertial referential (two observations in the inertial frame) $\left[ \vv{t_{2I}} \ \vv{t_{2K}} \ \vv{t_{3K}}
%\right]$~\cite{benet:swisscubeAttitudeDetermination,black:passiveAttitudeDetermination}.
%This is known as the triad frame:
%\begin{alignat}{4}
%    \vv{t_{1I}} &= \frac{\vv{v_1}}{\left| \vv{v_1} \right|} &\vv{t_{2I}} &{}={}&
%    \frac{\vv{u_1}}{\left| \vv{u_1} \right|} \ \ \ \ \ \ \  \\
%    \vv{t_{2I}} &= \frac{\vv{v_1} \times \vv{v_2}}{\left| \vv{v_1} \times \vv{v_2} \right|} \ \ \ \ \ \ \ \
%        &\vv{t_{2K}} &{}={}& \frac{\vv{u_1} \times \vv{u_2}}{\left| \vv{u_1} \times \vv{u_2} \right|} \\
%    \vv{t_{3I}} &= \vv{t_{1I}} \times \vv{t_{2I}} &\vv{t_{3K}} &{}={}& \vv{t_{2I}} \times \vv{t_{2K}}
%\end{alignat}
%Getting from frame $\kFrame$ to $\iFrame$ now simplifies to multiplication of the triad frame base change matrices:
%\begin{equation}
%    A =
%    \begin{bmatrix}
%        \vv{t_{1K}} & \vv{t_{2K}} & \vv{t_{3K}}
%    \end{bmatrix}
%    \begin{bmatrix}
%        \vv{t_{1I}} & \vv{t_{2I}} & \vv{t_{3I}}
%    \end{bmatrix}^T
%\end{equation}

%\begin{subequations}
%Relative to our solar system, the majority of bright stars ($m \!<\! 6.0$, or visible from the Earth with the naked
%eye) do not visibly move.
%Relative to our solar system, the majority of stars visible from Earth with the naked eye do not visibly move.
For simplicity, we make the assumption here that all stars in $\kFrame$ are fixed and exist in an inertial frame
known as the \textit{Earth centered inertial} (ECI) frame.
The star vectors themselves come from astronomical catalogs, recorded as points lying on the celestial
sphere~\cite{tappe:starTrackerDevelopment}.
Two pieces of information are given here: right ascension $\alpha$ (equivalent to latitude on Earth) and
declination $\delta$ (equivalent to longitude).
$\vv{K_j}$ represents a point $\left( \alpha, \delta \right)$ in a 3D spherical frame with a fixed radius for all stars,
projected to 3D Cartesian space.
%    Representing some spherical point $(\alpha, \delta, r)$ in 3D Cartesian space involves the following:
%    \begin{align} \label{eq:sphereToCartesian}
%    x &= r \cos(\delta) \cos(\alpha) \\
%    y &= r \cos(\delta) \sin(\alpha) \\
%    z &= r \sin(\delta)
%    \end{align}
%    where both $\alpha$ and $\delta$ are in degrees, and $r$ represents some constant distance from Earth.
%    $\vv{K_j}$ represents a point obtained from a star catalog that lies in the ECI frame, $r$ units away from Earth.
%\end{subequations}

Let $\vv{I_j}$ represent a 3D point projected from a 2D observation taken by the star tracker.
A basic star tracker is composed of a camera, a computer for determining orientation, and a link back to the main
computer.
After taking the picture, the pixel positions of potential stars in the image are determined.
This involves finding bright blobs in the image, and computing each blob's center of mass to get a point ($x, y$).
Through some 2D to 3D transformation process involving the camera's lens structure, a 3D point is then
obtained~\cite{tappe:starTrackerDevelopment}.
%    To align these stars with the ones in the catalog, the inverse Mercator mapping is used [CITE ME]:
%    \begin{align}
%        x &= k \cos\left( \frac{x}{R}  \right) \cos\left(2 * \arctan\left(\exp\left(\frac{y}{R}\right)\right) -
%            \frac{\pi}{2}\right) \\
%        y &= k \cos\left( \frac{x}{R}  \right) \sin\left(2 * \arctan\left(\exp\left(\frac{y}{R}\right)\right) -
%            \frac{\pi}{2}\right) \\
%        z &= k \sin\left( \frac{x}{R}  \right)
%    \end{align}

The next issue is the focus of this paper: determining which observation from the star tracker frame $\iFrame$
maps to which observation from the star catalog frame $\kFrame$.
Once this correspondence is found, Wahba's problem is solved to obtain $A$ and this is returned to the main computer.
