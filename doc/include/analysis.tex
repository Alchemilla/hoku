\section{Analysis}\label{sec:analysis}

\subsection{Feature Uniqueness}\label{subsec:featureUniquenessAnalysis}
Recall that $\sigma_1, \sigma_2, \ldots \sigma_n$ are hyperparameters that must be defined prior to executing the
algorithm.
The grid search exhausts every permutation $P_n^{\sigma_n}$, where:
\begin{equation}
    \label{eq:gridSearchEpsilon}
    \sigma_n \ \in \ \{1.0e^{-1}, 1.0e^{-2}, \ldots 1.0e^{-15}\}
\end{equation}

The set of plots in ADASDSA depict the effect on different $\sigma$ terms against the size of the candidate set $R$
(on the left) and the probability that the correct stars exist in this candidate set (on the right).
For simplicity, each query was restricted to $|R|=30$ candidates, each experiment was restricted to $N=25$ samples, and
every feature's $\sigma$ was set equal to all other $\sigma$s used in the query.

The most optimal $\sigma$ (largest $\sigma$ against the number of query hits) terms for each method are presented
below.
\begin{align*}
    \text{Angle}&: \sigma_\theta &= 1.0 \times 10^{-6}\\
    \text{Pyramid}& \sigma_\theta &= 1.0 \times 10^{-4}\\
    \text{Dot Angle}&: \sigma_{\theta_{ic}} &= 1.0 \times 10^{-4}\\
    \text{Dot Angle}&: \sigma_{\theta_{jc}} &= 1.0 \times 10^{-4}\\
    \text{Dot Angle}&: \sigma_\phi &= 1.0 \times 10^{-4} \\
    \text{Spherical Triangle}&: \sigma_a &= 1.0 \times 10^{-4}\\
    \text{Spherical Triangle}&: \sigma_\imath &= 1.0 \times 10^{-4}\\
    \text{Planar Triangle / Composite}&: \sigma_a &= 1.0 \times 10^{-4} \\
    \text{Planar Triangle / Composite}&: \sigma_\imath &= 1.0 \times 10^{-4}\\
\end{align*}

In ADADSDASDSAD, noise is plotted against the size of the candidate set $R$ (on the left) and the probability that the
correct stars exist in this candidate set (on the right).
Given the

The first set of plots show the deviation of noise against the size of candidate set $R$.

What is plotted in PUTGRAPHHERE depicts

\subsection{Candidate Reduction}\label{subsec:candidateReductionAnalysis}


\subsection{Alignment Determination}\label{subsec:alignmentDeterminationAnalysis}
In addition to the $\epsilon_n$ parameters used in the previous experiments, a parameter $\sigma_o$ must be defined
prior to executing the \Call{DMT}{} method.
This method is used by the Angle, Spherical Triangle, Planar Triangle, and Composite Pyramid methods.
The heuristic used to determine this value was a linear search, where:
\begin{equation}
    \label{eq:gridSearchSigma}
    \sigma_o \ \in \ \{1.0e^{-1}, 1.0e^{-2}, \ldots 1.0e^{-10}\}
\end{equation}
Each experiment was performed 10 times, and the largest $\sigma_o$ that identification the most stars correctly on
average are displayed below.
These were used for the experiment presented here.
\begin{align*}
    \text{Angle}&: \sigma_o &= 1.0 \times 10^{-4}\\
    \text{Spherical Triangle}&: \sigma_o &= 1.0 \times 10^{-4}\\
    \text{Planar Triangle}&: \sigma_o &= 1.0 \times 10^{-4}\\
    \text{Composite Pyramid}&: \sigma_o &= 1.0 \times 10^{-4}\\
\end{align*}
