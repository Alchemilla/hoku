\section{Related Work}\label{sec:relatedWork}
This section serves to give a brief overview into the different approaches to the lost-in-space star identification
problem.
Many of the methods listed here have also been described and compared by
Spratling~\cite{spratling:surveyStarIdentification} and Br\"{a}tt~\cite{bratt:analysisStarIdentification}.

\subsection{Subgraph Isomorphism Approach}\label{subsec:subgraphIsomorphismApproach}
\textit{Subgraph isomorphism} is a NP complete problem which aims to find some 1-to-1 mapping between the vertices
(stars) in two graphs (i.e.\ the catalog and the image) if it exists~\cite{scott:graphIsomorphismProblem}.
Methods that approach the star identification problem as such are the ones analyzed in this paper.

In 1978, Gottlieb developed the Polygon Angular Matching method which identifies 3 stars using interstar
angles~\cite{gottlieb:spacecraftAttitudeDetermination}.
In 1986, Groth introduced a Pattern Matching algorithm for lists of coordinates, which identifies 3 stars using the
relative ratios between each interstar angle and logarithm of this sum~\cite{groth:patternMatchingMethod}.
In 1991, Anderson proposed the use of storing permutations instead of combinations to avoid an additional
identification step at the expense of storage~\cite{spratling:surveyStarIdentification,anderson:autonomousStarSensing}.
In 1992, Renken developed a method which allowed for variable number of image subsets based on interstar angle matching
of 2 to 6 stars~\cite{bratt:analysisStarIdentification,renken:starConstellationMatching}.

In 1993, Baldini introduced a method which identifies $b$ bright stars in an image using five stars and their interstar
angles~\cite{spratling:surveyStarIdentification,baldini:starConstellationMatching}.
In 1994, Scholl proposed using the intesities themselves as features to remove the additional identification step
~\cite{spratling:surveyStarIdentification,scholl:starFieldIdentification}.
In 1995, Liebe developed the Liebe Star ID method which identifies 3 stars using two interstar angles and the
interior angle with one of the stars as the vertex~\cite{liebe:starTrackersAttitudeDetermination}.
Also in 1995, Ketchum developed the second sequential filtering algorithm, which identifies 2 stars using their
brightness~\cite{spratling:surveyStarIdentification,ketchum:onboardStarIdentification}.

In 2004, Mortari first developed the Pyramid method which introduces a special process for determining image star
subsets, a verification step to identify 4 stars, and a "searchless" database access
method to speed up query time~\cite{motari:pyramidIdentification}.
In 2004 and 2006, Cole \& Crassidus developed the Spherical and Planar Triangle methods
respectively which identifies 3 stars using the area and torque formed by the triangles of interstar angles, and a
process for reducing candidates~\cite{coleAndCrassidis:sphericalTriangleMethod, coleAndCrassidis:planarTriangleMethod}.
In 2005, Rousseau et. al introduced the use of the sine of the interior angle of three stars as a feature as well as
the idea of truncating the possible catalog matches based on a computed
attitude~\cite{bratt:analysisStarIdentification,rousseau:starRecognitionAPS}.

In 2007, Kolomenkin et. al developed the voting algorithm, which would accumulate votes for every combination of
two stars in the image if the interstar angle of some pair in the image matched the angle in the
catalog and maps stars based on the number of votes
received~\cite{bratt:analysisStarIdentification,kolomenkin:geometricVoting}.
In 2011, Tichy et. al developed the two-star voting algorithm, which votes in a similar fashion to the voting algorithm
but applies votes toward a single star based off two interstar angles instead of votes toward two stars
~\cite{bratt:analysisStarIdentification,tichy:preliminaryTestsCommericalImagers}.
In 2014, Toloei developed the Novel Stars ID method, which retains all of the key aspects of the Pyramid method but
uses the features from Planar Triangle method~\cite{toloei:compositeIdentification}.

\newcommand{\srightarrow}{\! \rightarrow \!}
\begin{figure}[ht]
    \centering{
    % Style for process block.
\tikzstyle{process} = [rectangle, text width=3cm, minimum width=3cm, minimum height=1cm,text centered, draw=black,
fill=orange!30]

% Style for terminal block.
\tikzstyle{terminal} = [rectangle, text width=1.7cm, minimum width=1.7cm, minimum height=1.7cm,text centered,
draw=black, fill=red!30]

% Style for decision block.
\tikzstyle{decision} = [diamond, text width=2cm, minimum width=2.5cm, minimum height=2.5cm,text centered, draw=black,
fill=green!30, inner sep=-10pt]

% Style for line.
\tikzstyle{line} = [draw, -latex']

\begin{tikzpicture}[node distance=1.8cm]
    \node[scale=1](getImage)[terminal]{Get Camera Image};
    \node[scale=1](pickQueryStars) [process, left of=getImage, xshift = -1.7cm] {Pick $k$ Image Stars};
    \node[scale=1](searchCatalog)[process, below of=pickQueryStars] {Search Catalog};
    \node[scale=1](confidentInCatalog)[decision, below of=searchCatalog, yshift=-0.5cm] {$|R| > 0$?};
    \node[scale=1](filterCandidates)[process, below of=confidentInCatalog, yshift=-0.5cm] {Filter Candidates};
    \node[scale=1](confidentAfterFilter)[decision, below of=filterCandidates, yshift=-0.5cm] {Confident in $r$?};
    \node[scale=1](findMap)[process, below of=confidentAfterFilter, yshift=-0.5cm]{Identify};
    \node[scale=1](confidentInMap)[decision, below of=findMap, yshift=-0.5cm] {Confident in $a$?};
    \node[scale=1](returnMap)[terminal, right of=confidentInMap, xshift = 1.7cm] {Return Map};

    \draw[->,>=stealth](getImage) -- node[scale=1.3, yshift=-0.3cm]{$I$}(pickQueryStars);
    \draw[->,>=stealth] (pickQueryStars) -- node[scale=1.3, xshift=0.5cm]{$b$}(searchCatalog);
    \draw[->, >=stealth] (searchCatalog) -- node[scale=1.3, xshift=0.5cm, yshift=-0.15cm]{$R$}(confidentInCatalog);
    \draw[->, >=stealth] (confidentInCatalog) -- node[anchor=east, yshift=0.1cm]{Yes}(filterCandidates);
    \draw[->, >=stealth] (filterCandidates) -- node[scale=1.3, xshift=0.5cm, yshift=-0.15cm]{$r$}(confidentAfterFilter);
    \draw[->, >=stealth] (confidentInMap) -- node[xshift=-0.05cm, yshift=0.2cm]{Yes}(returnMap);
    \draw[->, >=stealth] (confidentAfterFilter) -- node[anchor=east, yshift=0.1cm]{Yes} (findMap);
    \draw[->, >=stealth] (findMap) -- node[scale=1.3, xshift=0.5cm, yshift=-0.15cm]{$a$}(confidentInMap);

    \draw[->, >=stealth] (confidentInCatalog.west) -- ++(-1.1cm, 0cm) node[anchor=south, xshift=0.5cm]{No}
    |- (pickQueryStars.west);
    \draw[->, >=stealth] (confidentAfterFilter.west) -- ++(-1.1cm, 0cm) node[anchor=south, xshift=0.5cm]{No}
    |- (pickQueryStars.west);
    \draw[->, >=stealth](confidentInMap.west) -- ++(-1.1cm, 0cm) node[anchor=south, xshift=0.5cm]{No}
    |- (pickQueryStars.west);
\end{tikzpicture}
    \caption{
    Flowchart depicting the unified identification framework which all methods here follow.
    Given an image $I$, this process returns a bijection $h$ between some subset of the input $b$ and a subset of the
    catalog $r$.
    In the event all subsets are exhausted, the function $h: b \srightarrow \emptyset$ is returned (not depicted).
    } \label{figure:unifiedIdentificationFlowchart}
    }
\end{figure}

\subsection{Pattern Recognition Approaches}\label{subsec:otherApproaches}
The \textit{pattern recognition} approach differs from the subgraph ismorphism approach in that stars themselves exist
as a part of patterns to be matched across the catalog and the image~\cite{bittanti:starIdentificationStudy}.
This is the other main approach to the identification problem, most of which is composed of grid algorithms
~\cite{lee:modifiedGridAlgorithm,padgett:gridAlgorithm} and neural network
approaches~\cite{lindsey:neuralNetworkMethods,alvelda:neuralNetworkStar}.
%There exist several methods outside of the norm that appear to perform well, the most notable being Quan's Adaptive
%Ant Colony method~\cite{}.