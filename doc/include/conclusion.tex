\section{Conclusion}\label{sec:conclusion}
In this paper, we discussed six star identification methods and their strengths and weaknesses.
Each method was modified from their original literature to fit a general identification flow.
Portions that were interchangeable amongst all methods such as database access and centroid determination were
normalized or removed to focus on the star identification aspect itself.
To control the severity of our error, artifical images were generated.

The Angle method is the simplest of the six and has the fastest query step, but its runtime is heavily impacted by
the $\abs{R} = 1$ criterion and \Call{DMT}{} process.
The Dot Angle method's $f$ accuracy is the least sensitive to varying levels of Gaussian noise,
$K$ catalog and has the slowest query step.
The Planar Triangle and Spherical Triangle methods produced the smallest $R$ set after querying, but their $f$ accuracy
is the most sensitive to increasing amounts of false stars.
The Pyramid method is the fastest method given varying amounts of Gaussian noise \& false stars and is also the most
accurate given varying amounts of spikes, but is not able to achieve $100\%$ average accuracy due to its query step.
The Composite Pyramid method does not suffer from this inaccuracy problem, but does not achieve the same consistent
performance of the Pyramid or the triangle methods due to the number of filters implemented.

Overall, the Pyramid method handles both Gaussian noise and false stars the best in a reasonable amount of time.
