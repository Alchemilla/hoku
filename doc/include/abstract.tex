%\begin{abstract}
%    The process of identifying stars is integral toward stellar based orientation determination in spacecraft.
%    Star identification involves matching points in an image of the sky with stars in an astronomical catalog.
%    A unified framework for identification was created and used to analyze six variations of methods based on their
%    approach to star set identification, obtaining a single image to catalog star set match, and uniquely mapping
%    each star in a image star set to a catalog star set.
%    Each method was presented an artificial image, and aspects that were interchangeable among each process were
%    normalized.
%    Given an image with false stars, the Pyramid method has the highest average accuracy and is the fastest of the six.
%    Given an image where each star's true position is distributed randomly (Gaussian noise), the Spherical Triangle
%    method's accuracy is the least sensitive.
%\end{abstract}

\begin{abstract}
    Given a set of query points within an image coordinate system, constellation queries find the matching points in a
    database of known points within a standard coordinate system.
    Constellation queries are an integral part of orientation determination systems used in spacecrafts to orient and
    navigate themselves.
    The query points are bright spots in an image captured by a camera on the spacecraft and the database contains known
    celestial objects in a celestial coordinate system.
    This paper studies six existing constellation query processing strategies (Angle, Interior Angle, Spherical Triangle,
    Planar Triangle, Pyramid, Composite Pyramid) using a unified algorithmic framework and presents an extensive
    experimental evaluation of the six strategies.
    Given an image with false points, the Pyramid strategy has the highest average accuracy and is the fastest of the six.
    Given an image where each point's true position is distributed randomly (Gaussian noise), the Spherical
    Triangle strategy's accuracy is the least sensitive to positional change.
\end{abstract}
