\newcommand{\nsubparagraph}[1]{\subsubsection{#1}}

\section{Related Work}\label{sec:relatedWork}
This section serves to give a brief overview of the different approaches to querying for constellations.
More comprehensive survey papers have been published by Spratling~\cite{spratling:surveyStarIdentification} and
Br\"{a}tt~\cite{bratt:analysisStarIdentification}.

\nsubparagraph{Identification Classes}
The first main class of identification and the focus of this paper is the \textit{subgraph isomorphism} class.
Subgraph isomorphism is NP complete problem which aims to find some 1-to-1 mapping between the vertices (stars) in two
graphs (i.e.\ the database and the image) if it exists~\cite{scott:graphIsomorphismProblem}.
This involves describing and mapping sets of stars between both the database and image in terms of their features
relative to each other.

The second class of identification is the \textit{pattern recognition} class.
In contrast to subgraph isomorphism class, the pattern recognition class commonly deals with larger star sets within
some defined field-of-view and matches patterns rather than features.
Pattern formation typically involves 2D binary matrices (grids), where `1' occupies a cell with a star and `0' occupies
a cell without one~\cite{padgett:gridAlgorithm}.

%
%
%Another notable attempt toward pattern formation utlizes Delaunay Triangulation, as seen from Miri~\cite
%
%Some notable approaches utilize Padgett's~\cite{padgett:gridAlgorithm} and Lee's~\cite{lee:modifiedGridAlgorithm}
%use of binary matrices (grids) to construct patterns, Lindsey~\cite{lindsey:neuralNetworkMethods} and
%Alvelda's~\cite{alvelda:neuralNetworkStar} use of neural networks to optimize pattern similarity, and
%Paladugu's~\cite{paladugu:geneticAlgorithms} use of a genetic algorithm to solve the same pattern similarity problem.

\nsubparagraph{Recursive Property}
Recall that the lost-in-space condition specifies that we do not have any information about the spacecraft's attitude
prior to starting our identification algorithm.
For the majority of a star tracker's lifetime though, this constraint can be relaxed to allow for the use of
\textit{recursive} star identification.
Recursive strategies possess an attitude recorded at time $t$, and perform the identification at a later time $t + dt$.
Two strategies proposed by Samaan (SP-Search and SNA) reduce the amount of candidate stars from the database that could
map to stars from the image~\cite{samaan:recursiveMode}.

\nsubparagraph{Features}
Each star has a position associated with it, be it from a star database or from the image.
Using this position, the most common feature is the interstar angle between two stars, first utilized by Gottlieb to
identify sets of three stars with three angles~\cite{gottlieb:spacecraftAttitudeDetermination}.
Notable strategies with geometric functions utilizing these interstar angles were proposed by:
Groth~\cite{groth:patternMatchingMethod}, Cole \&
Crassidus~\cite{coleAndCrassidis:sphericalTriangleMethod,coleAndCrassidis:planarTriangleMethod}, and
Lang~\cite{lang:astrometryDotNet}.
Another common feature is the interior angle between three stars, where one star exists as a vertex to two other stars.
Liebe uses this in conjunction with interstar angles~\cite{liebe:starTrackersAttitudeDetermination}.

%Each star has a position associated with it, be it from a star catalog or from the image.
%Using this position, the most common feature is the interstar angle between two stars, first utilized by Gottlieb to
%identify sets of three stars with three angles~\cite{gottlieb:spacecraftAttitudeDetermination}.
%Additional geometric functions utilizing these interstar angles were proposed by: Groth (using the log of the sum of
%three interstar angles in a trio~\cite{groth:patternMatchingMethod}), Cole \& Crassidus (treating the angles of a
%star trio as sides to a triangle, and computing the triangle's area and
%torque)~\cite{coleAndCrassidis:sphericalTriangleMethod, coleAndCrassidis:planarTriangleMethod}, and Lang (using the
%differences in interstar angle between a star quad in a localized coordinate system)~\cite{lang:astrometryDotNet}.
%Another common feature is the interior angle between three stars, where one star exists as a vertex to two other stars.
%Liebe uses this in conjuction with interstar angles~\cite{liebe:starTrackersAttitudeDetermination}.

%This feature is used solely by Rousseau (the sine of the closest two closest stars of a trio)
%~\cite{rousseau:starRecognitionAPS}
%and Samaan (the interior angles of three stars)
%~\cite{samaan:nondimensionalStarIdentification},
%and in conjunction with the interstar angles by Liebe~\cite{liebe:starTrackersAttitudeDetermination}.

Each star also has a brightness attached it, a feature less commonly used due to large variance in measurement.
Spratling describes two early strategies to take advantage of this feature.
Scholl proposed the usage of this to remove the need for ambiguity after matching star subsets with angular features
~\cite{scholl:starFieldIdentification}.
Ketchum later introduced the second sequential filtering algorithm, which identifies two stars using their brightness
in comparison to the common trio required of interstar angle strategies~\cite{ketchum:onboardStarIdentification}.
More recent work toward integrating brightness more heavily has been performed by Zhang et
al~\cite{zhang:brightnessReferenced}.

\nsubparagraph{Database Access}
The naive approach to searching for matching features in a subgraph isomorphism approach is to perform a
linear search across the entire database and search for matching subsets.
Early star identification strategies focused on reducing the size of the database to be searched, rather than the search
process itself.
In 1996, Quine (according to Spratling) was the first to reduce the database search time from linear to log time
using a binary search tree~\cite{quine:fastAutonomousStarAcquistion}.
The following year Mortari's "Search-Less Algorithm" was introduced, which utilizes $k$-vectors to search the database
independent of its size~\cite{mortari:kVectorApproach}.

\newcommand{\srightarrow}{\! \rightarrow \!}
\begin{figure}
    \centering{
    % Style for process block.
\tikzstyle{process} = [rectangle, text width=3cm, minimum width=3cm, minimum height=1cm,text centered, draw=black,
fill=orange!30]

% Style for terminal block.
\tikzstyle{terminal} = [rectangle, text width=1.7cm, minimum width=1.7cm, minimum height=1.7cm,text centered,
draw=black, fill=red!30]

% Style for decision block.
\tikzstyle{decision} = [diamond, text width=2cm, minimum width=2.5cm, minimum height=2.5cm,text centered, draw=black,
fill=green!30, inner sep=-10pt]

% Style for line.
\tikzstyle{line} = [draw, -latex']

\begin{tikzpicture}[node distance=1.8cm]
    \node[scale=1](getImage)[terminal]{Get Camera Image};
    \node[scale=1](pickQueryStars) [process, left of=getImage, xshift = -1.7cm] {Pick $k$ Image Stars};
    \node[scale=1](searchCatalog)[process, below of=pickQueryStars] {Search Catalog};
    \node[scale=1](confidentInCatalog)[decision, below of=searchCatalog, yshift=-0.5cm] {$|R| > 0$?};
    \node[scale=1](filterCandidates)[process, below of=confidentInCatalog, yshift=-0.5cm] {Filter Candidates};
    \node[scale=1](confidentAfterFilter)[decision, below of=filterCandidates, yshift=-0.5cm] {Confident in $r$?};
    \node[scale=1](findMap)[process, below of=confidentAfterFilter, yshift=-0.5cm]{Identify};
    \node[scale=1](confidentInMap)[decision, below of=findMap, yshift=-0.5cm] {Confident in $a$?};
    \node[scale=1](returnMap)[terminal, right of=confidentInMap, xshift = 1.7cm] {Return Map};

    \draw[->,>=stealth](getImage) -- node[scale=1.3, yshift=-0.3cm]{$I$}(pickQueryStars);
    \draw[->,>=stealth] (pickQueryStars) -- node[scale=1.3, xshift=0.5cm]{$b$}(searchCatalog);
    \draw[->, >=stealth] (searchCatalog) -- node[scale=1.3, xshift=0.5cm, yshift=-0.15cm]{$R$}(confidentInCatalog);
    \draw[->, >=stealth] (confidentInCatalog) -- node[anchor=east, yshift=0.1cm]{Yes}(filterCandidates);
    \draw[->, >=stealth] (filterCandidates) -- node[scale=1.3, xshift=0.5cm, yshift=-0.15cm]{$r$}(confidentAfterFilter);
    \draw[->, >=stealth] (confidentInMap) -- node[xshift=-0.05cm, yshift=0.2cm]{Yes}(returnMap);
    \draw[->, >=stealth] (confidentAfterFilter) -- node[anchor=east, yshift=0.1cm]{Yes} (findMap);
    \draw[->, >=stealth] (findMap) -- node[scale=1.3, xshift=0.5cm, yshift=-0.15cm]{$a$}(confidentInMap);

    \draw[->, >=stealth] (confidentInCatalog.west) -- ++(-1.1cm, 0cm) node[anchor=south, xshift=0.5cm]{No}
    |- (pickQueryStars.west);
    \draw[->, >=stealth] (confidentAfterFilter.west) -- ++(-1.1cm, 0cm) node[anchor=south, xshift=0.5cm]{No}
    |- (pickQueryStars.west);
    \draw[->, >=stealth](confidentInMap.west) -- ++(-1.1cm, 0cm) node[anchor=south, xshift=0.5cm]{No}
    |- (pickQueryStars.west);
\end{tikzpicture}
    \caption{
    Flowchart depicting the unified identification framework which all strategies here follow.
    Given an image $I$, this process returns a function $h$ whose domain is some subset of the image $b$ and whose
    codomain is a subset of the stellar database, $r$.
    In the event all subsets are exhausted, the function $h: b \srightarrow \emptyset$ is returned (not depicted).
    } \label{figure:unifiedIdentificationFlowchart}
    }
\end{figure}

\nsubparagraph{Mapping}
To identify a star in an image is to pair it with some star in a database.
Gottlieb's strategy used a voting approach to remove the ambiguity after identifying a single star
pair~\cite{gottlieb:spacecraftAttitudeDetermination},
which was later generalized by Kolomenkin to vote for every star in the image~\cite{kolomenkin:geometricVoting}.
The direct match test was proposed by Needleman (according to Tappe), which determines the likelihood of a map based
on how many stars from each frame align with the attitude formed by the map~\cite{needelman:stellarAttitudeAcquisition}.
In an effort to avoid the mapping processes above, Anderson (according to Spratling) proposed the use of storing
permutations of star subsets instead of combinations at the expense of storage~\cite{anderson:autonomousStarSensing}.
The use of neural networks~\cite{lindsey:neuralNetworkMethods,alvelda:neuralNetworkStar} and genetic
algorithms~\cite{paladugu:geneticAlgorithms} have also been proposed to optimize the mapping process.
